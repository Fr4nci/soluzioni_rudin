\documentclass{report}
%%%%%%%%%%%%%%%%%%%%%%%%%%%%%%%%%
% PACKAGE IMPORTS
%%%%%%%%%%%%%%%%%%%%%%%%%%%%%%%%%


\usepackage[tmargin=2cm,rmargin=1.5in,lmargin=1.5in,margin=0.85in,bmargin=2cm,footskip=.2in]{geometry}
\usepackage{bookmark}
\usepackage{amsmath,amsfonts,amsthm,amssymb,mathtools}
\usepackage[varbb]{newpxmath}
\usepackage{chemfig}
\usepackage{xfrac}
\usepackage{mhchem}
\usepackage[italian]{babel}
\usepackage[makeroom]{cancel}
\usepackage{mathtools}
\usepackage{listing}
\usepackage{enumitem}
\usepackage{hyperref,theoremref}
\hypersetup{
	pdftitle={Soluzioni Rudin},
	colorlinks=true, linkcolor=doc!90,
	bookmarksnumbered=true,
	bookmarksopen=true
}
\usepackage[most,many,breakable]{tcolorbox}
\usepackage{xcolor}
\usepackage{varwidth}
\usepackage{varwidth}
\usepackage{etoolbox}
%\usepackage{authblk}
\usepackage{nameref}
\usepackage{multicol,array}
\usepackage{tikz-cd}
\usepackage[ruled,vlined,linesnumbered]{algorithm2e}
\usepackage{comment} % enables the use of multi-line comments (\ifx \fi) 
\usepackage{import}
\usepackage{xifthen}
\usepackage{pdfpages}
\usepackage{transparent}

\newcommand\mycommfont[1]{\footnotesize\ttfamily\textcolor{blue}{#1}}
\SetCommentSty{mycommfont}
\newcommand{\incfig}[1]{%
    \def\svgwidth{\columnwidth}
    \import{./figures/}{#1.pdf_tex}
}

\usepackage{tikzsymbols}
\usepackage{float}
\renewcommand\qedsymbol{$\square$}
\usepackage{hyperref}

%\usepackage{import}
%\usepackage{xifthen}
%\usepackage{pdfpages}
%\usepackage{transparent}


%%%%%%%%%%%%%%%%%%%%%%%%%%%%%%
% SELF MADE COLORS
%%%%%%%%%%%%%%%%%%%%%%%%%%%%%%



\definecolor{myg}{RGB}{56, 140, 70}
\definecolor{myb}{RGB}{45, 111, 177}
\definecolor{myr}{RGB}{199, 68, 64}
\definecolor{mytheorembg}{HTML}{F2F2F9}
\definecolor{mytheoremfr}{HTML}{00007B}
\definecolor{mylenmabg}{HTML}{FFFAF8}
\definecolor{mylenmafr}{HTML}{983b0f}
\definecolor{mypropbg}{HTML}{f2fbfc}
\definecolor{mypropfr}{HTML}{191971}
\definecolor{myexamplebg}{HTML}{F2FBF8}
\definecolor{myexamplefr}{HTML}{88D6D1}
\definecolor{myexampleti}{HTML}{2A7F7F}
\definecolor{mydefinitbg}{HTML}{E5E5FF}
\definecolor{mydefinitfr}{HTML}{3F3FA3}
\definecolor{notesgreen}{RGB}{0,162,0}
\definecolor{myp}{RGB}{197, 92, 212}
\definecolor{mygr}{HTML}{2C3338}
\definecolor{myred}{RGB}{127,0,0}
\definecolor{myyellow}{RGB}{169,121,69}
\definecolor{myexercisebg}{HTML}{F2FBF8}
\definecolor{myexercisefg}{HTML}{88D6D1}


%%%%%%%%%%%%%%%%%%%%%%%%%%%%
% TCOLORBOX SETUPS
%%%%%%%%%%%%%%%%%%%%%%%%%%%%

\setlength{\parindent}{1cm}
%================================
% THEOREM BOX
%================================

\tcbuselibrary{theorems,skins,hooks}
\newtcbtheorem[number within=section]{Theorem}{Teorema}
{%
	enhanced,
	breakable,
	colback = mytheorembg,
	frame hidden,
	boxrule = 0sp,
	borderline west = {2pt}{0pt}{mytheoremfr},
	sharp corners,
	detach title,
	before upper = \tcbtitle\par\smallskip,
	coltitle = mytheoremfr,
	fonttitle = \bfseries\sffamily,
	description font = \mdseries,
	separator sign none,
	segmentation style={solid, mytheoremfr},
}
{th}

\tcbuselibrary{theorems,skins,hooks}
\newtcbtheorem[number within=chapter]{theorem}{Teorema}
{%
	enhanced,
	breakable,
	colback = mytheorembg,
	frame hidden,
	boxrule = 0sp,
	borderline west = {2pt}{0pt}{mytheoremfr},
	sharp corners,
	detach title,
	before upper = \tcbtitle\par\smallskip,
	coltitle = mytheoremfr,
	fonttitle = \bfseries\sffamily,
	description font = \mdseries,
	separator sign none,
	segmentation style={solid, mytheoremfr},
}
{th}


\tcbuselibrary{theorems,skins,hooks}
\newtcolorbox{Theoremcon}
{%
	enhanced
	,breakable
	,colback = mytheorembg
	,frame hidden
	,boxrule = 0sp
	,borderline west = {2pt}{0pt}{mytheoremfr}
	,sharp corners
	,description font = \mdseries
	,separator sign none
}

%================================
% Corollery
%================================
\tcbuselibrary{theorems,skins,hooks}
\newtcbtheorem[number within=section]{Corollary}{Corollario}
{%
	enhanced
	,breakable
	,colback = myp!10
	,frame hidden
	,boxrule = 0sp
	,borderline west = {2pt}{0pt}{myp!85!black}
	,sharp corners
	,detach title
	,before upper = \tcbtitle\par\smallskip
	,coltitle = myp!85!black
	,fonttitle = \bfseries\sffamily
	,description font = \mdseries
	,separator sign none
	,segmentation style={solid, myp!85!black}
}
{th}
\tcbuselibrary{theorems,skins,hooks}
\newtcbtheorem[number within=chapter]{corollary}{Corollario}
{%
	enhanced
	,breakable
	,colback = myp!10
	,frame hidden
	,boxrule = 0sp
	,borderline west = {2pt}{0pt}{myp!85!black}
	,sharp corners
	,detach title
	,before upper = \tcbtitle\par\smallskip
	,coltitle = myp!85!black
	,fonttitle = \bfseries\sffamily
	,description font = \mdseries
	,separator sign none
	,segmentation style={solid, myp!85!black}
}
{th}


%================================
% LENMA
%================================

\tcbuselibrary{theorems,skins,hooks}
\newtcbtheorem[number within=section]{Lenma}{Lemma}
{%
	enhanced,
	breakable,
	colback = mylenmabg,
	frame hidden,
	boxrule = 0sp,
	borderline west = {2pt}{0pt}{mylenmafr},
	sharp corners,
	detach title,
	before upper = \tcbtitle\par\smallskip,
	coltitle = mylenmafr,
	fonttitle = \bfseries\sffamily,
	description font = \mdseries,
	separator sign none,
	segmentation style={solid, mylenmafr},
}
{th}

\tcbuselibrary{theorems,skins,hooks}
\newtcbtheorem[number within=chapter]{lenma}{Lemma}
{%
	enhanced,
	breakable,
	colback = mylenmabg,
	frame hidden,
	boxrule = 0sp,
	borderline west = {2pt}{0pt}{mylenmafr},
	sharp corners,
	detach title,
	before upper = \tcbtitle\par\smallskip,
	coltitle = mylenmafr,
	fonttitle = \bfseries\sffamily,
	description font = \mdseries,
	separator sign none,
	segmentation style={solid, mylenmafr},
}
{th}


%================================
% PROPOSITION
%================================

\tcbuselibrary{theorems,skins,hooks}
\newtcbtheorem[number within=section]{Prop}{Proposizione}
{%
	enhanced,
	breakable,
	colback = mypropbg,
	frame hidden,
	boxrule = 0sp,
	borderline west = {2pt}{0pt}{mypropfr},
	sharp corners,
	detach title,
	before upper = \tcbtitle\par\smallskip,
	coltitle = mypropfr,
	fonttitle = \bfseries\sffamily,
	description font = \mdseries,
	separator sign none,
	segmentation style={solid, mypropfr},
}
{th}

\tcbuselibrary{theorems,skins,hooks}
\newtcbtheorem[number within=chapter]{prop}{Proposizione}
{%
	enhanced,
	breakable,
	colback = mypropbg,
	frame hidden,
	boxrule = 0sp,
	borderline west = {2pt}{0pt}{mypropfr},
	sharp corners,
	detach title,
	before upper = \tcbtitle\par\smallskip,
	coltitle = mypropfr,
	fonttitle = \bfseries\sffamily,
	description font = \mdseries,
	separator sign none,
	segmentation style={solid, mypropfr},
}
{th}


%================================
% CLAIM
%================================

\tcbuselibrary{theorems,skins,hooks}
\newtcbtheorem[number within=section]{claim}{Claim}
{%
	enhanced
	,breakable
	,colback = myg!10
	,frame hidden
	,boxrule = 0sp
	,borderline west = {2pt}{0pt}{myg}
	,sharp corners
	,detach title
	,before upper = \tcbtitle\par\smallskip
	,coltitle = myg!85!black
	,fonttitle = \bfseries\sffamily
	,description font = \mdseries
	,separator sign none
	,segmentation style={solid, myg!85!black}
}
{th}



%================================
% Exercise
%================================

\tcbuselibrary{theorems,skins,hooks}
\newtcbtheorem[number within=section]{Exercise}{Esercizio}
{%
	enhanced,
	breakable,
	colback = myexercisebg,
	frame hidden,
        oversize,
	boxrule = 0sp,
	borderline west = {2pt}{0pt}{myexercisefg},
	sharp corners,
	detach title,
	before upper = \tcbtitle\par\smallskip,
	coltitle = myexercisefg,
	fonttitle = \bfseries\sffamily,
	description font = \mdseries,
	separator sign none,
	segmentation style={solid, myexercisefg},
}
{th}

\tcbuselibrary{theorems,skins,hooks}
\newtcbtheorem[number within=chapter]{exercise}{Esercizio}
{%
	enhanced,
	breakable,
	colback = myexercisebg,
	frame hidden,
        oversize,
	boxrule = 0sp,
	borderline west = {2pt}{0pt}{myexercisefg},
	sharp corners,
	detach title,
	before upper = \tcbtitle\par\smallskip,
	coltitle = myexercisefg,
	fonttitle = \bfseries\sffamily,
	description font = \mdseries,
	separator sign none,
	segmentation style={solid, myexercisefg},
}
{th}
%================================
% EXAMPLE BOX
%================================

\newtcbtheorem[number within=section]{Example}{Esempio}
{%
	colback = myexamplebg
	,breakable
	,colframe = myexamplefr
	,coltitle = myexampleti
	,boxrule = 1pt
	,sharp corners
	,detach title
	,before upper=\tcbtitle\par\smallskip
	,fonttitle = \bfseries
	,description font = \mdseries
	,separator sign none
	,description delimiters parenthesis
}
{ex}

\newtcbtheorem[number within=chapter]{example}{Esempio}
{%
	colback = myexamplebg
	,breakable
	,colframe = myexamplefr
	,coltitle = myexampleti
	,boxrule = 1pt
	,sharp corners
	,detach title
	,before upper=\tcbtitle\par\smallskip
	,fonttitle = \bfseries
	,description font = \mdseries
	,separator sign none
	,description delimiters parenthesis
}
{ex}

%================================
% DEFINITION BOX
%================================

\newtcbtheorem[number within=section]{Definition}{Definizione}{enhanced,
	before skip=2mm,after skip=2mm, colback=red!5,colframe=red!80!black,boxrule=0.5mm,
	attach boxed title to top left={xshift=1cm,yshift*=1mm-\tcboxedtitleheight}, varwidth boxed title*=-3cm,
	boxed title style={frame code={
					\path[fill=tcbcolback]
					([yshift=-1mm,xshift=-1mm]frame.north west)
					arc[start angle=0,end angle=180,radius=1mm]
					([yshift=-1mm,xshift=1mm]frame.north east)
					arc[start angle=180,end angle=0,radius=1mm];
					\path[left color=tcbcolback!60!black,right color=tcbcolback!60!black,
						middle color=tcbcolback!80!black]
					([xshift=-2mm]frame.north west) -- ([xshift=2mm]frame.north east)
					[rounded corners=1mm]-- ([xshift=1mm,yshift=-1mm]frame.north east)
					-- (frame.south east) -- (frame.south west)
					-- ([xshift=-1mm,yshift=-1mm]frame.north west)
					[sharp corners]-- cycle;
				},interior engine=empty,
		},
	fonttitle=\bfseries,
	title={#2},#1}{def}
\newtcbtheorem[number within=chapter]{definition}{Definizione}{enhanced,
	before skip=2mm,after skip=2mm, colback=red!5,colframe=red!80!black,boxrule=0.5mm,
	attach boxed title to top left={xshift=1cm,yshift*=1mm-\tcboxedtitleheight}, varwidth boxed title*=-3cm,
	boxed title style={frame code={
					\path[fill=tcbcolback]
					([yshift=-1mm,xshift=-1mm]frame.north west)
					arc[start angle=0,end angle=180,radius=1mm]
					([yshift=-1mm,xshift=1mm]frame.north east)
					arc[start angle=180,end angle=0,radius=1mm];
					\path[left color=tcbcolback!60!black,right color=tcbcolback!60!black,
						middle color=tcbcolback!80!black]
					([xshift=-2mm]frame.north west) -- ([xshift=2mm]frame.north east)
					[rounded corners=1mm]-- ([xshift=1mm,yshift=-1mm]frame.north east)
					-- (frame.south east) -- (frame.south west)
					-- ([xshift=-1mm,yshift=-1mm]frame.north west)
					[sharp corners]-- cycle;
				},interior engine=empty,
		},
	fonttitle=\bfseries,
	title={#2},#1}{def}



%================================
% Question BOX
%================================

\makeatletter
\newtcbtheorem{question}{Domanda}{enhanced,
	breakable,
	colback=white,
	colframe=myb!80!black,
	attach boxed title to top left={yshift*=-\tcboxedtitleheight},
	fonttitle=\bfseries,
	title={#2},
	boxed title size=title,
	boxed title style={%
			sharp corners,
			rounded corners=northwest,
			colback=tcbcolframe,
			boxrule=0pt,
		},
	underlay boxed title={%
			\path[fill=tcbcolframe] (title.south west)--(title.south east)
			to[out=0, in=180] ([xshift=5mm]title.east)--
			(title.center-|frame.east)
			[rounded corners=\kvtcb@arc] |-
			(frame.north) -| cycle;
		},
	#1
}{def}
\makeatother

%================================
% SOLUTION BOX
%================================

\makeatletter
\newtcolorbox{solution}{enhanced,
	breakable,
	colback=white,
	colframe=myg!80!black,
	attach boxed title to top left={yshift*=-\tcboxedtitleheight},
	title=Soluzione,
	boxed title size=title,
	boxed title style={%
			sharp corners,
			rounded corners=northwest,
			colback=tcbcolframe,
			boxrule=0pt,
		},
	underlay boxed title={%
			\path[fill=tcbcolframe] (title.south west)--(title.south east)
			to[out=0, in=180] ([xshift=5mm]title.east)--
			(title.center-|frame.east)
			[rounded corners=\kvtcb@arc] |-
			(frame.north) -| cycle;
		},
}
\makeatother

%================================
% Question BOX
%================================

\makeatletter
\newtcbtheorem{qstion}{Risposta}{enhanced,
	breakable,
	colback=white,
	colframe=mygr,
	attach boxed title to top left={yshift*=-\tcboxedtitleheight},
	fonttitle=\bfseries,
	title={#2},
	boxed title size=title,
	boxed title style={%
			sharp corners,
			rounded corners=northwest,
			colback=tcbcolframe,
			boxrule=0pt,
		},
	underlay boxed title={%
			\path[fill=tcbcolframe] (title.south west)--(title.south east)
			to[out=0, in=180] ([xshift=5mm]title.east)--
			(title.center-|frame.east)
			[rounded corners=\kvtcb@arc] |-
			(frame.north) -| cycle;
		},
	#1
}{def}
\makeatother

\newtcbtheorem[number within=chapter]{wconc}{Wrong Concept}{
	breakable,
	enhanced,
	colback=white,
	colframe=myr,
	arc=0pt,
	outer arc=0pt,
	fonttitle=\bfseries\sffamily\large,
	colbacktitle=myr,
	attach boxed title to top left={},
	boxed title style={
			enhanced,
			skin=enhancedfirst jigsaw,
			arc=3pt,
			bottom=0pt,
			interior style={fill=myr}
		},
	#1
}{def}



%================================
% NOTE BOX
%================================

\usetikzlibrary{arrows,calc,shadows.blur}
\tcbuselibrary{skins}
\newtcolorbox{note}[1][]{%
	enhanced jigsaw,
	colback=gray!20!white,%
	colframe=gray!80!black,
	size=small,
	boxrule=1pt,
	title=\textbf{Oss:-},
	halign title=flush center,
	coltitle=black,
	breakable,
	drop shadow=black!50!white,
	attach boxed title to top left={xshift=1cm,yshift=-\tcboxedtitleheight/2,yshifttext=-\tcboxedtitleheight/2},
	minipage boxed title=1.5cm,
	boxed title style={%
			colback=white,
			size=fbox,
			boxrule=1pt,
			boxsep=2pt,
			underlay={%
					\coordinate (dotA) at ($(interior.west) + (-0.5pt,0)$);
					\coordinate (dotB) at ($(interior.east) + (0.5pt,0)$);
					\begin{scope}
						\clip (interior.north west) rectangle ([xshift=3ex]interior.east);
						\filldraw [white, blur shadow={shadow opacity=60, shadow yshift=-.75ex}, rounded corners=2pt] (interior.north west) rectangle (interior.south east);
					\end{scope}
					\begin{scope}[gray!80!black]
						\fill (dotA) circle (2pt);
						\fill (dotB) circle (2pt);
					\end{scope}
				},
		},
	#1,
}

%%%%%%%%%%%%%%%%%%%%%%%%%%%%%%
% SELF MADE COMMANDS
%%%%%%%%%%%%%%%%%%%%%%%%%%%%%%


\newcommand{\thm}[2]{\begin{theorem}{#1}{}#2\end{theorem}}
\newcommand{\cor}[2]{\begin{corollary}{#1}{}#2\end{corollary}}
\newcommand{\mlenma}[2]{\begin{lenma}{#1}{}#2\end{lenma}}
\newcommand{\mprop}[2]{\begin{prop}{#1}{}#2\end{prop}}
\newcommand{\clm}[3]{\begin{claim}{#1}{#2}#3\end{claim}}
\newcommand{\wc}[2]{\begin{wconc}{#1}{}\setlength{\parindent}{1cm}#2\end{wconc}}
\newcommand{\thmcon}[1]{\begin{Theoremcon}{#1}\end{Theoremcon}}
\newcommand{\ex}[2]{\begin{example}{#1}{}#2\end{example}}
\newcommand{\exc}[2]{\begin{Exercise}{#1}{}#2\end{Exercise}}
%\newcommand{\dfn}[2]{\begin{Definition}[colbacktitle=red!75!black]{#1}{}#2\end{Definition}}
\newcommand{\dfn}[2]{\begin{definition}[colbacktitle=red!75!black]{#1}{}#2\end{definition}}
\newcommand{\qs}[2]{\begin{question}{#1}{}#2\end{question}}
\newcommand{\pf}[2]{\begin{myproof}[#1]#2\end{myproof}}
\newcommand{\nt}[1]{\begin{note}#1\end{note}}

\newcommand*\circled[1]{\tikz[baseline=(char.base)]{
		\node[shape=circle,draw,inner sep=1pt] (char) {#1};}}
\newcommand\getcurrentref[1]{%
	\ifnumequal{\value{#1}}{0}
	{??}
	{\the\value{#1}}%
}
\newcommand{\getCurrentSectionNumber}{\getcurrentref{section}}
\newenvironment{myproof}[1][\proofname]{%
	\proof[\bfseries #1: ]%
}{\endproof}

\newcommand{\mclm}[2]{\begin{myclaim}[#1]#2\end{myclaim}}
\newenvironment{myclaim}[1][\claimname]{\proof[\bfseries #1: ]}{}

\newcounter{mylabelcounter}

\makeatletter
\newcommand{\setword}[2]{%
	\phantomsection
	#1\def\@currentlabel{\unexpanded{#1}}\label{#2}%
}
\makeatother




\tikzset{
	symbol/.style={
			draw=none,
			every to/.append style={
					edge node={node [sloped, allow upside down, auto=false]{$#1$}}}
		}
}


% deliminators
\DeclarePairedDelimiter{\abs}{\lvert}{\rvert}
\DeclarePairedDelimiter{\norm}{\lVert}{\rVert}

\DeclarePairedDelimiter{\ceil}{\lceil}{\rceil}
\DeclarePairedDelimiter{\floor}{\lfloor}{\rfloor}
\DeclarePairedDelimiter{\round}{\lfloor}{\rceil}

\newsavebox\diffdbox
\newcommand{\slantedromand}{{\mathpalette\makesl{d}}}
\newcommand{\makesl}[2]{%
\begingroup
\sbox{\diffdbox}{$\mathsurround=0pt#1\mathrm{#2}$}%
\pdfsave
\pdfsetmatrix{1 0 0.2 1}%
\rlap{\usebox{\diffdbox}}%
\pdfrestore
\hskip\wd\diffdbox
\endgroup
}
\newcommand{\dd}[1][]{\ensuremath{\mathop{}\!\ifstrempty{#1}{%
\slantedromand\@ifnextchar^{\hspace{0.2ex}}{\hspace{0.1ex}}}%
{\slantedromand\hspace{0.2ex}^{#1}}}}
\ProvideDocumentCommand\dv{o m g}{%
  \ensuremath{%
    \IfValueTF{#3}{%
      \IfNoValueTF{#1}{%
        \frac{\dd #2}{\dd #3}%
      }{%
        \frac{\dd^{#1} #2}{\dd #3^{#1}}%
      }%
    }{%
      \IfNoValueTF{#1}{%
        \frac{\dd}{\dd #2}%
      }{%
        \frac{\dd^{#1}}{\dd #2^{#1}}%
      }%
    }%
  }%
}
\providecommand*{\pdv}[3][]{\frac{\partial^{#1}#2}{\partial#3^{#1}}}
%  - others
\DeclareMathOperator{\Lap}{\mathcal{L}}
\DeclareMathOperator{\Var}{Var} % varience
\DeclareMathOperator{\Cov}{Cov} % covarience
\DeclareMathOperator{\E}{E} % expected

% Since the amsthm package isn't loaded

% I prefer the slanted \leq
\let\oldleq\leq % save them in case they're every wanted
\let\oldgeq\geq
\renewcommand{\leq}{\leqslant}
\renewcommand{\geq}{\geqslant}

% % redefine matrix env to allow for alignment, use r as default
% \renewcommand*\env@matrix[1][r]{\hskip -\arraycolsep
%     \let\@ifnextchar\new@ifnextchar
%     \array{*\c@MaxMatrixCols #1}}


%\usepackage{framed}
%\usepackage{titletoc}
%\usepackage{etoolbox}
%\usepackage{lmodern}


%\patchcmd{\tableofcontents}{\contentsname}{\sffamily\contentsname}{}{}

%\renewenvironment{leftbar}
%{\def\FrameCommand{\hspace{6em}%
%		{\color{myyellow}\vrule width 2pt depth 6pt}\hspace{1em}}%
%	\MakeFramed{\parshape 1 0cm \dimexpr\textwidth-6em\relax\FrameRestore}\vskip2pt%
%}
%{\endMakeFramed}

%\titlecontents{chapter}
%[0em]{\vspace*{2\baselineskip}}
%{\parbox{4.5em}{%
%		\hfill\Huge\sffamily\bfseries\color{myred}\thecontentspage}%
%	\vspace*{-2.3\baselineskip}\leftbar\textsc{\small\chaptername~\thecontentslabel}\\\sffamily}
%{}{\endleftbar}
%\titlecontents{section}
%[8.4em]
%{\sffamily\contentslabel{3em}}{}{}
%{\hspace{0.5em}\nobreak\itshape\color{myred}\contentspage}
%\titlecontents{subsection}
%[8.4em]
%{\sffamily\contentslabel{3em}}{}{}  
%{\hspace{0.5em}\nobreak\itshape\color{myred}\contentspage}



%%%%%%%%%%%%%%%%%%%%%%%%%%%%%%%%%%%%%%%%%%%
% TABLE OF CONTENTS
%%%%%%%%%%%%%%%%%%%%%%%%%%%%%%%%%%%%%%%%%%%

\usepackage{tikz}
\definecolor{doc}{RGB}{0,0,0}
\definecolor{pink_flyod}{RGB}{252, 49, 153}
\usepackage{titletoc}
\contentsmargin{0cm}
\titlecontents{chapter}[3.7pc]
{\addvspace{30pt}%
	\begin{tikzpicture}[remember picture, overlay]%
		\draw[fill=doc!60,draw=doc!60] (-7,-.1) rectangle (-0.9,.5);%
		\pgftext[left,x=-3.5cm,y=0.2cm]{\color{white}\Large\sc\bfseries Capitolo\ \thecontentslabel};%
	\end{tikzpicture}\color{doc!60}\large\sc\bfseries}%
{}
{}
{\;\titlerule\;\large\sc\bfseries Pagina \thecontentspage
	\begin{tikzpicture}[remember picture, overlay]
		\draw[fill=doc!60,draw=doc!60] (2pt,0) rectangle (4,0.1pt);
	\end{tikzpicture}}%
\titlecontents{section}[3.7pc]
{\addvspace{2pt}}
{\contentslabel[\thecontentslabel]{2pc}}
{}
{\hfill\small \thecontentspage}
[]
\titlecontents*{subsection}[3.7pc]
{\addvspace{-1pt}\small}
{}
{}
{\ --- \small\thecontentspage}
[ \textbullet\ ][]

\makeatletter
\renewcommand{\tableofcontents}{%
	\chapter*{%
	  \vspace*{-20\p@}%
	  \begin{tikzpicture}[remember picture, overlay]%
		  \pgftext[right,x=15cm,y=0.2cm]{\color{doc!60}\Huge\sc\bfseries \contentsname};%
		  \draw[fill=doc!60,draw=doc!60] (13,-.75) rectangle (20,1);%
		  \clip (13,-.75) rectangle (20,1);
		  \pgftext[right,x=15cm,y=0.2cm]{\color{white}\Huge\sc\bfseries \contentsname};%
	  \end{tikzpicture}}%
	\@starttoc{toc}}
\makeatother

%%%%%%%%%%%%%%%%%%%%%%%%
% QUOTE
%%%%%%%%%%%%%%%%%%%%%%%%
\newtcolorbox{zitat}[1][]{%
    after={\par\smallskip\noindent},
    colback=white,
    grow to right by=-10mm,
    grow to left by=-10mm, 
    boxrule=0pt,
    boxsep=0pt,
    breakable,
    enhanced jigsaw,
    borderline west={0.5pt}{0pt}{black},
    colbacktitle={white},
    coltitle={black},
    fonttitle={\large\bfseries},
}
\newcommand{\qte}[1]{\begin{zitat} #1 \end{zitat}}

%%%%%%%%%%%%%%%%%%%%%%%%%
% LSTLISTINGS COLORS
%%%%%%%%%%%%%%%%%%%%%%%%%
\definecolor{codegreen}{rgb}{0,0.6,0}
\definecolor{codegray}{rgb}{0.5,0.5,0.5}
\definecolor{codepurple}{rgb}{0.58,0,0.82}
\definecolor{backcolour}{rgb}{0.95,0.95,0.92}

\lstdefinestyle{code}{
    backgroundcolor=\color{backcolour},   
    commentstyle=\color{codegreen},
    keywordstyle=\color{magenta},
    numberstyle=\tiny\color{codegray},
    stringstyle=\color{codepurple},
    basicstyle=\ttfamily\footnotesize,
    breakatwhitespace=false,         
    breaklines=true,                 
    captionpos=b,                    
    keepspaces=true,                                     
    showspaces=false,                
    showstringspaces=false,
    showtabs=false,                  
    tabsize=2
}

\lstset{style=code}
%From M275 "Topology" at SJSU
\newcommand{\id}{\mathrm{id}}
\newcommand{\taking}[1]{\xrightarrow{#1}}
\newcommand{\inv}{^{-1}}

%From M170 "Introduction to Graph Theory" at SJSU
\DeclareMathOperator{\diam}{diam}
\DeclareMathOperator{\ord}{ord}
\newcommand{\defeq}{\overset{\mathrm{def}}{=}}

%From the USAMO .tex files
\newcommand{\ts}{\textsuperscript}
\newcommand{\dg}{^\circ}
\newcommand{\ii}{\item}

% % From Math 55 and Math 145 at Harvard
% \newenvironment{subproof}[1][Proof]{%
% \begin{proof}[#1] \renewcommand{\qedsymbol}{$\blacksquare$}}%
% {\end{proof}}

\newcommand{\liff}{\leftrightarrow}
\newcommand{\lthen}{\rightarrow}
\newcommand{\opname}{\operatorname}
\newcommand{\surjto}{\twoheadrightarrow}
\newcommand{\injto}{\hookrightarrow}
\newcommand{\On}{\mathrm{On}} % ordinals
\DeclareMathOperator{\img}{im} % Image
\DeclareMathOperator{\Img}{Im} % Image
\DeclareMathOperator{\coker}{coker} % Cokernel
\DeclareMathOperator{\Coker}{Coker} % Cokernel
\DeclareMathOperator{\Ker}{Ker} % Kernel
\DeclareMathOperator{\rank}{rank}
\DeclareMathOperator{\Spec}{Spec} % spectrum
\DeclareMathOperator{\Tr}{Tr} % trace
\DeclareMathOperator{\pr}{pr} % projection
\DeclareMathOperator{\ext}{ext} % extension
\DeclareMathOperator{\pred}{pred} % predecessor
\DeclareMathOperator{\dom}{dom} % domain
\DeclareMathOperator{\ran}{ran} % range
\DeclareMathOperator{\Hom}{Hom} % homomorphism
\DeclareMathOperator{\Mor}{Mor} % morphisms
\DeclareMathOperator{\End}{End} % endomorphism

\newcommand{\eps}{\epsilon}
\newcommand{\veps}{\varepsilon}
\newcommand{\ol}{\overline}
\newcommand{\ul}{\underline}
\newcommand{\wt}{\widetilde}
\newcommand{\wh}{\widehat}
\newcommand{\vocab}[1]{\textbf{\color{blue} #1}}
\providecommand{\half}{\frac{1}{2}}
\newcommand{\dang}{\measuredangle} %% Directed angle
\newcommand{\ray}[1]{\overrightarrow{#1}}
\newcommand{\seg}[1]{\overline{#1}}
\newcommand{\arc}[1]{\wideparen{#1}}
\DeclareMathOperator{\cis}{cis}
\DeclareMathOperator*{\lcm}{lcm}
\DeclareMathOperator*{\argmin}{arg min}
\DeclareMathOperator*{\argmax}{arg max}
\newcommand{\cycsum}{\sum_{\mathrm{cyc}}}
\newcommand{\symsum}{\sum_{\mathrm{sym}}}
\newcommand{\cycprod}{\prod_{\mathrm{cyc}}}
\newcommand{\symprod}{\prod_{\mathrm{sym}}}
\newcommand{\Qed}{\begin{flushright}\qed\end{flushright}}
\newcommand{\parinn}{\setlength{\parindent}{1cm}}
\newcommand{\parinf}{\setlength{\parindent}{0cm}}
% \newcommand{\norm}{\|\cdot\|}
\newcommand{\inorm}{\norm_{\infty}}
\newcommand{\opensets}{\{V_{\alpha}\}_{\alpha\in I}}
\newcommand{\oset}{V_{\alpha}}
\newcommand{\opset}[1]{V_{\alpha_{#1}}}
\newcommand{\lub}{\text{lub}}
\newcommand{\del}[2]{\frac{\partial #1}{\partial #2}}
\newcommand{\Del}[3]{\frac{\partial^{#1} #2}{\partial^{#1} #3}}
\newcommand{\deld}[2]{\dfrac{\partial #1}{\partial #2}}
\newcommand{\Deld}[3]{\dfrac{\partial^{#1} #2}{\partial^{#1} #3}}
\newcommand{\lm}{\lambda}
\newcommand{\uin}{\mathbin{\rotatebox[origin=c]{90}{$\in$}}}
\newcommand{\usubset}{\mathbin{\rotatebox[origin=c]{90}{$\subset$}}}
\newcommand{\lt}{\left}
\newcommand{\rt}{\right}
\newcommand{\bs}[1]{\boldsymbol{#1}}
\newcommand{\exs}{\exists}
\newcommand{\st}{\strut}
\newcommand{\dps}[1]{\displaystyle{#1}}

\newcommand{\sol}{\setlength{\parindent}{0cm}\textbf{\textit{Soluzione:}}\setlength{\parindent}{1cm} }
\newcommand{\solve}[1]{\setlength{\parindent}{0cm}\textbf{\textit{Soluzione: }}\setlength{\parindent}{1cm}#1 \Qed}

% Things Lie
\newcommand{\kb}{\mathfrak b}
\newcommand{\kg}{\mathfrak g}
\newcommand{\kh}{\mathfrak h}
\newcommand{\kn}{\mathfrak n}
\newcommand{\ku}{\mathfrak u}
\newcommand{\kz}{\mathfrak z}
\DeclareMathOperator{\Ext}{Ext} % Ext functor
\DeclareMathOperator{\Tor}{Tor} % Tor functor
\newcommand{\gl}{\opname{\mathfrak{gl}}} % frak gl group
\renewcommand{\sl}{\opname{\mathfrak{sl}}} % frak sl group chktex 6

% More script letters etc.
\newcommand{\SA}{\mathcal A}
\newcommand{\SB}{\mathcal B}
\newcommand{\SC}{\mathcal C}
\newcommand{\SF}{\mathcal F}
\newcommand{\SG}{\mathcal G}
\newcommand{\SH}{\mathcal H}
\newcommand{\OO}{\mathcal O}

\newcommand{\SCA}{\mathscr A}
\newcommand{\SCB}{\mathscr B}
\newcommand{\SCC}{\mathscr C}
\newcommand{\SCD}{\mathscr D}
\newcommand{\SCE}{\mathscr E}
\newcommand{\SCF}{\mathscr F}
\newcommand{\SCG}{\mathscr G}
\newcommand{\SCH}{\mathscr H}

% Mathfrak primes
\newcommand{\km}{\mathfrak m}
\newcommand{\kp}{\mathfrak p}
\newcommand{\kq}{\mathfrak q}

% number sets
\newcommand{\RR}[1][]{\ensuremath{\ifstrempty{#1}{\mathbb{R}}{\mathbb{R}^{#1}}}}
\newcommand{\NN}[1][]{\ensuremath{\ifstrempty{#1}{\mathbb{N}}{\mathbb{N}^{#1}}}}
\newcommand{\ZZ}[1][]{\ensuremath{\ifstrempty{#1}{\mathbb{Z}}{\mathbb{Z}^{#1}}}}
\newcommand{\QQ}[1][]{\ensuremath{\ifstrempty{#1}{\mathbb{Q}}{\mathbb{Q}^{#1}}}}
\newcommand{\CC}[1][]{\ensuremath{\ifstrempty{#1}{\mathbb{C}}{\mathbb{C}^{#1}}}}
\newcommand{\PP}[1][]{\ensuremath{\ifstrempty{#1}{\mathbb{P}}{\mathbb{P}^{#1}}}}
\newcommand{\HH}[1][]{\ensuremath{\ifstrempty{#1}{\mathbb{H}}{\mathbb{H}^{#1}}}}
\newcommand{\FF}[1][]{\ensuremath{\ifstrempty{#1}{\mathbb{F}}{\mathbb{F}^{#1}}}}
% expected value
\newcommand{\EE}{\ensuremath{\mathbb{E}}}
\newcommand{\charin}{\text{ char }}
\DeclareMathOperator{\sign}{sign}
\DeclareMathOperator{\Aut}{Aut}
\DeclareMathOperator{\Inn}{Inn}
\DeclareMathOperator{\Syl}{Syl}
\DeclareMathOperator{\Gal}{Gal}
\DeclareMathOperator{\GL}{GL} % General linear group
\DeclareMathOperator{\SL}{SL} % Special linear group

%---------------------------------------
% BlackBoard Math Fonts :-
%---------------------------------------

%Captital Letters
\newcommand{\bbA}{\mathbb{A}}	\newcommand{\bbB}{\mathbb{B}}
\newcommand{\bbC}{\mathbb{C}}	\newcommand{\bbD}{\mathbb{D}}
\newcommand{\bbE}{\mathbb{E}}	\newcommand{\bbF}{\mathbb{F}}
\newcommand{\bbG}{\mathbb{G}}	\newcommand{\bbH}{\mathbb{H}}
\newcommand{\bbI}{\mathbb{I}}	\newcommand{\bbJ}{\mathbb{J}}
\newcommand{\bbK}{\mathbb{K}}	\newcommand{\bbL}{\mathbb{L}}
\newcommand{\bbM}{\mathbb{M}}	\newcommand{\bbN}{\mathbb{N}}
\newcommand{\bbO}{\mathbb{O}}	\newcommand{\bbP}{\mathbb{P}}
\newcommand{\bbQ}{\mathbb{Q}}	\newcommand{\bbR}{\mathbb{R}}
\newcommand{\bbS}{\mathbb{S}}	\newcommand{\bbT}{\mathbb{T}}
\newcommand{\bbU}{\mathbb{U}}	\newcommand{\bbV}{\mathbb{V}}
\newcommand{\bbW}{\mathbb{W}}	\newcommand{\bbX}{\mathbb{X}}
\newcommand{\bbY}{\mathbb{Y}}	\newcommand{\bbZ}{\mathbb{Z}}

%---------------------------------------
% MathCal Fonts :-
%---------------------------------------

%Captital Letters
\newcommand{\mcA}{\mathcal{A}}	\newcommand{\mcB}{\mathcal{B}}
\newcommand{\mcC}{\mathcal{C}}	\newcommand{\mcD}{\mathcal{D}}
\newcommand{\mcE}{\mathcal{E}}	\newcommand{\mcF}{\mathcal{F}}
\newcommand{\mcG}{\mathcal{G}}	\newcommand{\mcH}{\mathcal{H}}
\newcommand{\mcI}{\mathcal{I}}	\newcommand{\mcJ}{\mathcal{J}}
\newcommand{\mcK}{\mathcal{K}}	\newcommand{\mcL}{\mathcal{L}}
\newcommand{\mcM}{\mathcal{M}}	\newcommand{\mcN}{\mathcal{N}}
\newcommand{\mcO}{\mathcal{O}}	\newcommand{\mcP}{\mathcal{P}}
\newcommand{\mcQ}{\mathcal{Q}}	\newcommand{\mcR}{\mathcal{R}}
\newcommand{\mcS}{\mathcal{S}}	\newcommand{\mcT}{\mathcal{T}}
\newcommand{\mcU}{\mathcal{U}}	\newcommand{\mcV}{\mathcal{V}}
\newcommand{\mcW}{\mathcal{W}}	\newcommand{\mcX}{\mathcal{X}}
\newcommand{\mcY}{\mathcal{Y}}	\newcommand{\mcZ}{\mathcal{Z}}


%---------------------------------------
% Bold Math Fonts :-
%---------------------------------------

%Captital Letters
\newcommand{\bmA}{\boldsymbol{A}}	\newcommand{\bmB}{\boldsymbol{B}}
\newcommand{\bmC}{\boldsymbol{C}}	\newcommand{\bmD}{\boldsymbol{D}}
\newcommand{\bmE}{\boldsymbol{E}}	\newcommand{\bmF}{\boldsymbol{F}}
\newcommand{\bmG}{\boldsymbol{G}}	\newcommand{\bmH}{\boldsymbol{H}}
\newcommand{\bmI}{\boldsymbol{I}}	\newcommand{\bmJ}{\boldsymbol{J}}
\newcommand{\bmK}{\boldsymbol{K}}	\newcommand{\bmL}{\boldsymbol{L}}
\newcommand{\bmM}{\boldsymbol{M}}	\newcommand{\bmN}{\boldsymbol{N}}
\newcommand{\bmO}{\boldsymbol{O}}	\newcommand{\bmP}{\boldsymbol{P}}
\newcommand{\bmQ}{\boldsymbol{Q}}	\newcommand{\bmR}{\boldsymbol{R}}
\newcommand{\bmS}{\boldsymbol{S}}	\newcommand{\bmT}{\boldsymbol{T}}
\newcommand{\bmU}{\boldsymbol{U}}	\newcommand{\bmV}{\boldsymbol{V}}
\newcommand{\bmW}{\boldsymbol{W}}	\newcommand{\bmX}{\boldsymbol{X}}
\newcommand{\bmY}{\boldsymbol{Y}}	\newcommand{\bmZ}{\boldsymbol{Z}}
%Small Letters
\newcommand{\bma}{\boldsymbol{a}}	\newcommand{\bmb}{\boldsymbol{b}}
\newcommand{\bmc}{\boldsymbol{c}}	\newcommand{\bmd}{\boldsymbol{d}}
\newcommand{\bme}{\boldsymbol{e}}	\newcommand{\bmf}{\boldsymbol{f}}
\newcommand{\bmg}{\boldsymbol{g}}	\newcommand{\bmh}{\boldsymbol{h}}
\newcommand{\bmi}{\boldsymbol{i}}	\newcommand{\bmj}{\boldsymbol{j}}
\newcommand{\bmk}{\boldsymbol{k}}	\newcommand{\bml}{\boldsymbol{l}}
\newcommand{\bmm}{\boldsymbol{m}}	\newcommand{\bmn}{\boldsymbol{n}}
\newcommand{\bmo}{\boldsymbol{o}}	\newcommand{\bmp}{\boldsymbol{p}}
\newcommand{\bmq}{\boldsymbol{q}}	\newcommand{\bmr}{\boldsymbol{r}}
\newcommand{\bms}{\boldsymbol{s}}	\newcommand{\bmt}{\boldsymbol{t}}
\newcommand{\bmu}{\boldsymbol{u}}	\newcommand{\bmv}{\boldsymbol{v}}
\newcommand{\bmw}{\boldsymbol{w}}	\newcommand{\bmx}{\boldsymbol{x}}
\newcommand{\bmy}{\boldsymbol{y}}	\newcommand{\bmz}{\boldsymbol{z}}

%---------------------------------------
% Scr Math Fonts :-
%---------------------------------------

\newcommand{\sA}{{\mathscr{A}}}   \newcommand{\sB}{{\mathscr{B}}}
\newcommand{\sC}{{\mathscr{C}}}   \newcommand{\sD}{{\mathscr{D}}}
\newcommand{\sE}{{\mathscr{E}}}   \newcommand{\sF}{{\mathscr{F}}}
\newcommand{\sG}{{\mathscr{G}}}   \newcommand{\sH}{{\mathscr{H}}}
\newcommand{\sI}{{\mathscr{I}}}   \newcommand{\sJ}{{\mathscr{J}}}
\newcommand{\sK}{{\mathscr{K}}}   \newcommand{\sL}{{\mathscr{L}}}
\newcommand{\sM}{{\mathscr{M}}}   \newcommand{\sN}{{\mathscr{N}}}
\newcommand{\sO}{{\mathscr{O}}}   \newcommand{\sP}{{\mathscr{P}}}
\newcommand{\sQ}{{\mathscr{Q}}}   \newcommand{\sR}{{\mathscr{R}}}
\newcommand{\sS}{{\mathscr{S}}}   \newcommand{\sT}{{\mathscr{T}}}
\newcommand{\sU}{{\mathscr{U}}}   \newcommand{\sV}{{\mathscr{V}}}
\newcommand{\sW}{{\mathscr{W}}}   \newcommand{\sX}{{\mathscr{X}}}
\newcommand{\sY}{{\mathscr{Y}}}   \newcommand{\sZ}{{\mathscr{Z}}}


%---------------------------------------
% Math Fraktur Font
%---------------------------------------

%Captital Letters
\newcommand{\mfA}{\mathfrak{A}}	\newcommand{\mfB}{\mathfrak{B}}
\newcommand{\mfC}{\mathfrak{C}}	\newcommand{\mfD}{\mathfrak{D}}
\newcommand{\mfE}{\mathfrak{E}}	\newcommand{\mfF}{\mathfrak{F}}
\newcommand{\mfG}{\mathfrak{G}}	\newcommand{\mfH}{\mathfrak{H}}
\newcommand{\mfI}{\mathfrak{I}}	\newcommand{\mfJ}{\mathfrak{J}}
\newcommand{\mfK}{\mathfrak{K}}	\newcommand{\mfL}{\mathfrak{L}}
\newcommand{\mfM}{\mathfrak{M}}	\newcommand{\mfN}{\mathfrak{N}}
\newcommand{\mfO}{\mathfrak{O}}	\newcommand{\mfP}{\mathfrak{P}}
\newcommand{\mfQ}{\mathfrak{Q}}	\newcommand{\mfR}{\mathfrak{R}}
\newcommand{\mfS}{\mathfrak{S}}	\newcommand{\mfT}{\mathfrak{T}}
\newcommand{\mfU}{\mathfrak{U}}	\newcommand{\mfV}{\mathfrak{V}}
\newcommand{\mfW}{\mathfrak{W}}	\newcommand{\mfX}{\mathfrak{X}}
\newcommand{\mfY}{\mathfrak{Y}}	\newcommand{\mfZ}{\mathfrak{Z}}
%Small Letters
\newcommand{\mfa}{\mathfrak{a}}	\newcommand{\mfb}{\mathfrak{b}}
\newcommand{\mfc}{\mathfrak{c}}	\newcommand{\mfd}{\mathfrak{d}}
\newcommand{\mfe}{\mathfrak{e}}	\newcommand{\mff}{\mathfrak{f}}
\newcommand{\mfg}{\mathfrak{g}}	\newcommand{\mfh}{\mathfrak{h}}
\newcommand{\mfi}{\mathfrak{i}}	\newcommand{\mfj}{\mathfrak{j}}
\newcommand{\mfk}{\mathfrak{k}}	\newcommand{\mfl}{\mathfrak{l}}
\newcommand{\mfm}{\mathfrak{m}}	\newcommand{\mfn}{\mathfrak{n}}
\newcommand{\mfo}{\mathfrak{o}}	\newcommand{\mfp}{\mathfrak{p}}
\newcommand{\mfq}{\mathfrak{q}}	\newcommand{\mfr}{\mathfrak{r}}
\newcommand{\mfs}{\mathfrak{s}}	\newcommand{\mft}{\mathfrak{t}}
\newcommand{\mfu}{\mathfrak{u}}	\newcommand{\mfv}{\mathfrak{v}}
\newcommand{\mfw}{\mathfrak{w}}	\newcommand{\mfx}{\mathfrak{x}}
\newcommand{\mfy}{\mathfrak{y}}	\newcommand{\mfz}{\mathfrak{z}}

\title{Soluzioni agli esercizi di \emph{Principles of Mathematical Analysis} di W. Rudin}
\author{Francesco Sermi}
\date{\today}


\begin{document}
	\maketitle
	\tableofcontents
	\chapter{The real and complex number system}
	\begin{enumerate}[label=\protect\circled{\arabic*}]
		\item se $r$ è razionale ($r \neq 0$) e $x$ è irrazionale, provare che $r+x$ e $rx$ sono irrazionali
		\begin{myproof}
			Supponiamo per assurdo che $r \in \mathbb{Q}$ e $x$ sia irrazionale, mentre $r+x$ e $rx$ siano razionali. Allora, $r+x = \frac{m}{n}$ con $m, n \in \mathbb{Z}$. Ma siccome $r \in \mathbb{Q} \implies \exists p, q \in \mathbb{Z}: r = \frac{p}{q}$ e
			$$
				r + x = \frac{p}{q} + x = \frac{m}{n} \implies x = \frac{p}{q} - \frac{m}{n} = \frac{pn - qm}{qn} \implies x \in \mathbb{Q}
			$$ 
			il che è assurdo. \\
			Procediamo con $rx$ alla solita maniera: se $rx \in \mathbb{Q} \implies \exists m, n \in \mathbb{Z}: rx = \frac{m}{n}$.  Ma allora, sapendo che $r = \frac{p}{q}$ con $p, q \in \mathbb{Z}$ in virtù della sua razionalità, $x = \frac{m}{nr} = \frac{mq}{np} \implies x \in \mathbb{Q}$ il che è nuovamente assurdo.
		\end{myproof}
\end{enumerate}
\begin{enumerate}[resume, label=\protect\circled{\arabic*}]
		\item Provare che non esiste razionale $q$ tale che $q^2 = 12$
\end{enumerate}
		\begin{myproof}
			si osservi il seguente lemma (di cui non daremo dimostrazione)
			\mlenma{(di Euclide)}{Sia $n \in \mathbb{Z}$ e $n$ è primo.Se $n | ab$ e $a$ è coprimo con $b$ (o viceversa) allora $n | a \vee n | b$}
		\noindent Supponiamo per assurdo che $\exists q \in \mathbb{Q} : q^2 = 12$. Data la razionalità di $q$ abbiamo che esistono $m, n \in \mathbb{Z}: q = \frac{m}{n}$ e $m$ e $n$ coprimi fra allora. Ciò implica che:
		$$
			\frac{m}{n} = \sqrt{12} \implies \frac{m^2}{n^2} = 12 \implies m^2 = 12n^2
		$$
		Questo vuol dire che $m$ è pari. Siccome $2 | m \implies \exists k \in \mathbb{Z} : m = 2k$ e dunque
		$$
		m^2 = (2k)^2 = 4k^2 = 12n^2 \implies k^2 = 3n^2
		$$
		Siccome il lato destro è divisibile per $3$ allora si deve avere che anche il lato sinistro è divisibile per $3$ e dunque, per il lemma di Euclide, si osserva che si deve avere che $3 | k \implies \exists q \in \mathbb{Z} : k = 3q$. Si deduce che
		$$
		k^2 = 9q^2 = 3n^2 \implies n^2 = 3q^2
		$$
		dunque $n^2$ è divisibile per $3$ e, sempre per il lemma di Euclide, $n$ è divisibile per $3$. Ma allora si giunge ad un assurdo siccome $m = 2k$ con $3 | k$ e $3 | n$ contro l'ipotesi di coprimità fra $m$ e $n$
		\end{myproof}
		Un'ulteriore dimostrazione poteva essere effettuata basandosi sul fatto che $\sqrt{12} = 2\sqrt{3}$ dunque, tramite l'esercizio 1, sappiamo che $rx$ è irrazionale se $x$ è irrazionale e $r$ razionale quindi la dimostrazione si riduceva a provare che $\sqrt{3}$ è irrazionale.
\begin{enumerate}[resume, label=\protect\circled{\arabic*}]
		\item Provare la seguente proposizione
\end{enumerate}		
		\mprop{Conseguenze degli assiomi moltiplicativi di cui gode il campo $\mathbb{R}$}{Gli assiomi moltiplicativi di cui gode $\mathbb{R}$ implicano le seguenti proprietà:
		\begin{enumerate}[label=\protect\circled{\arabic*}]
			\item Se $x \neq 0$ e $xy = xz \implies y=z$
			\item Se $x \neq 0$ e $xy = x \implies y = 1$
			\item Se $x \neq 0$ e $xy = 1 \implies y = x^{-1}$
			\item Se $x \neq 0$ mostrare che $(x^{-1})^{-1} = x$
		\end{enumerate}}
	\begin{myproof}
	per dimostrare la \circled{1}, banalmente, si ha che:
	$$
	y \stackrel{\text{esistenza di un elemento inverso e } x \neq 0}{=}  x x^{-1} y \stackrel{\text{prop. commutativa}}{=} xz x^{-1} \stackrel{\text{prop. commutativa}}{=} x x^{-1} z = z \implies y=z
	$$
	La \circled{2} segue direttamente dalla prima ponendo $z=1$, così come la \circled{3} ponendo $z=x^{-1}$. Per la \circled{4} si osserva che siccome $\forall x \neq 0, x x^{-1} = 1$ allora $\frac{1}{x} (\frac{1}{x})^{-1} = 1 \implies x \frac{1}{x} (\frac{1}{x})^{-1} = x \implies (\frac{1}{x})^{-1} = x$
	\end{myproof}
\begin{enumerate}[resume, label=\protect\circled{\arabic*}]
	\item Sia $E \subset A$ con $A$ insieme ordinato (totalmente? Il Rudin non ce lo fa sapere ma è abbastanza probabile). Supponiamo che $\alpha$ sia un minorante di $E$ e $\beta$ sia un maggiorante di $E$. Provare che $\alpha \leq \beta$
\end{enumerate}
\begin{myproof}
	per definizione abbiamo che se $\alpha$ è un minorante allora $\forall x \in E, \alpha \leq x$ e se $\beta$ è un maggiorante allora $\forall x \in E, x \leq \beta$. Per transitività si ha che $\alpha \leq \beta$
\end{myproof}
\begin{enumerate}[resume, label=\protect\circled{\arabic*}]
	\item Sia A un insieme non vuoto di numeri reali che è limitato inferiormente. Sia $-A$ l'insieme di tutti i numeri $-x$, con $x \in A$. Mostrare che
	$$
		\inf{A} = - \sup{(-A)}
	$$
\end{enumerate}
\begin{myproof}
	sia $y \in \mathbb{R}$ un minorante di $A$. Allora si osserva che, per definizione, $\forall x \in A, y \leq x \implies -y \geq -x$ dunque $-A$ sarà limitato superiormente. Siccome $\forall E \subset \mathbb{R} \implies \exists \sup{E}, \inf{E} \in \mathbb{R}$ allora sappiamo che $-A$ avrà $\sup{(-A)} \in \mathbb{R}$ che denoteremo con $z = \sup{(-A)}$ e mostriamo la tesi, ovvero che $\sup{(-A)} = -\inf{A}$: dobbiamo mostrare che $-z$ è l'estremo inferiore. Per farlo si osserva che se $w > -z \implies z > -w$ dunque $-w$ non è un maggiorante di $-A$ dunque $\exists y=-x(x \in A \text{ per def.}) \in -A, z > y > -w \implies -z < -y < w$ ma siccome $-y=-(-x) = x \implies x < w$ dunque $w$ non è un minorante di $A$. Se invece supponiamo esista $w \in A : w < -z \implies -w > z$ ma $-w \in -A$ il che è assurdo siccome $\nexists w \in -A: w > \sup{(-A)}$. Dunque possiamo concludere che $\inf{A} = -\sup{(-A)}$ siccome abbiamo dimostrato che:
\begin{enumerate}[label=\protect\circled{\arabic*}]
	\item $-z$ è un minorante di A;
	\item $\forall x > -z \implies x $ non è un minorante 
\end{enumerate}
\end{myproof}
\begin{enumerate}[resume, label=\protect\circled{\arabic*}]
	\item Fissato $b>1$ \begin{enumerate}
		\item Se $m, n, p, q \in \mathbb{Z}, n>0, q>0$ e $r=\frac{m}{n} = \frac{p}{q}$ mostrare che
		$$
			(b^m)^{\frac{1}{n}} = (b^p)^{\frac{1}{q}}
 		$$
 		Dunque ha senso definire $b^r = (b^m)^{\frac{1}{n}}$
 		\item Mostrare che $b^{r+s} = b^r b^s$ se $r, s \in \mathbb{Q}$
 		\item Se $x$ è reale, definiamo $B(x)$ come l'insieme di tutti i numeri $b^t$, dove $t$ è un numero razionale e $t \leq x$. Mostrare che
 		$$
 			b^r = \sup{B(r)}
 		$$
 		dunque ha senso definire
 		$$
 		b^x = \sup{B(x)}
 		$$
 		$\forall x \in \mathbb{R}$
 		\item Mostrare che $b^{x+y}=b^x b^y \, \forall x, y \in \mathbb{R}$
	\end{enumerate}
\end{enumerate}
\begin{myproof}
	Per mostrare (a) si osserva che, dal teorema 1.21, si deve avere che esistono due numeri reali $r_1$ e $r_2$ che identificano univocamente $(b^m)^{\frac{1}{n}}$ e $(b^p)^{\frac{1}{q}}$ rispettivamente. La tesi dunque si ottiene mostrando che $r_1 = r_2$. Si osserva che $(r_1)^n = b^m$ e $(r_2)^q = b^p$ e, siccome $r$ è razionale, possiamo scrivere che $m = rn$ e $p = rq$. Dunque
	\begin{align*}
		&(r_1)^n = b^{rn} & &(r_2)^q = b^{rq}
	\end{align*}
	ma elevando la prima eguaglianza da entrambi le parti per $q$ e la seconda per $n$ si ottiene che 
	\begin{align*}
		(r_1)^{nq} = b^{rnq}, (r_2)^{nq} = b^{nqr} \implies r_1 = r_2
	\end{align*}
	Per mostrare la (b) si osserva che se $r,s \in \mathbb{Q}$ allora si ha che $\exists m, n, p, q \in \mathbb{Z}: r=\frac{m}{n} \, \text{e} \, s = \frac{p}{q}$. Dunque
	$$
	b^{r+s} = b^{\frac{m}{n} + \frac{p}{q}} = b^{\frac{mq + np}{nq}}
	$$
	come prima sappiamo che esisteranno $r_1$ e $r_2$ univocamente determinati tali che $r_1 = b^{r+s}$ e $r_2 = b^{r} b^{s}$ e vogliamo mostrare che $r_1 = r_2$. Adesso si osserva che
	$$
		(b^{r+s})^{nq} = b^{mq+np} = (r_1)^{nq}
	$$
	e
	$$
		(b^{r} b^s)^{nq} = (b^{r})^{nq} (b^s)^{nq} = (b^{\frac{m}{n}})^{nq} (b^{\frac{p}{q}})^{nq} = b^{mq} b^{pn} = b^{mq+np}
	$$
	dunque $(b^{r+s})^{nq} = (b^r b^s)^{nq} \implies b^{r+s} = b^r b^s$. \\ Per mostrare la (c) bisogna innanzitutto osservare che, definendo $B(x)$ come sopra si ha che, dati $t, s \in B(x), t \leq s \implies b^t \leq b^s$. Si può mostrare questo fatto in maniera abbastanza semplice ricordando come viene definita la relazione d'ordine $\leq$ sui razionali:
\mlenma{$b^x$ è crescente con $x \in \mathbb{Q}$}{$b^x$ ristretta ai razionali è una funzione crescente}
\begin{myproof}
	Supponiamo che $s, t \in \mathbb{Q}$ con $s \leq t$. Siccome $s$ e $t$ sono razionali, allora esisteranno $m, n, p, q \in \mathbb{Z}$ tali che $s = \frac{m}{n}$ e $t = \frac{p}{q}$. Dunque abbiamo che $s \leq t \iff mq \leq np$. Abbiamo che $b^s = (b^m)^{\frac{1}{n}}$ e $b^t = (b^p)^{\frac{1}{q}}$ e sappiamo che questi numeri identificano in maniera univoca, grazie al teorema 1.21, due distinti numeri reali (tranne nel caso in cui $s = t$). Si osserva che se $r_1 = (b^m)^{\frac{1}{n}} \implies (r_1)^{n} = b^m \implies (r_1)^{nq} = b^{mq} < b^{np} = (r_2)^{nq} \implies r_1 \leq r_2$  (prendere la radice non cambia la direzione della disuguaglianza siccome possiamo sfruttare l'identità $b^n - a^n = (b-a)\sum\limits_{i=1}^{n} b^{n-i} a^{i-1}$ e osservare che $b^n \geq a^n \iff b \geq a$) 
\end{myproof}	
	
\noindent Se consideriamo $B(r)$ con $r$ razionale, allora si ha banalmente che $b^r \in B(r)$ e dev'essere l'estremo superiore: infatti, se consideriamo $t > r \implies b^t > b^r$, dunque $b^t > b^r \geq b^x \implies b^t > b^x \, \forall x: b^x \in B(r)$ dunque $t$ è un maggiorante di $B(r)$. Mostriamo che $\forall b^x < b^r: b^x$ non è un maggiorante di $B(r)$: se per assurdo $\exists \alpha < b^r: \alpha$ è maggiorante, allora $\forall b^y \in B(r), b^y \leq \alpha < b^r$ il che è assurdo siccome $b^r \in B(r)$ e avremmo che $b^r \leq \alpha < b^r \implies b^r < b^r$. \\
	Per mostrare la (d) si osserva che dobbiamo mostrare, per il punto precedente, che $\sup{B(x+y)} = \sup{B(x)}\sup{B(y)} = b^x b^y$. Per fare ciò faremo uso del seguente lemma
	\mlenma{Unicità del $\sup$ e $\inf$}{Sia $A \subseteq \mathbb{R}$ un insieme non vuoto limitato superiormente (inferiormente). Allora $\sup{A}$ ($\inf{A}$) esiste ed è unico}
	\begin{myproof}
	l'esistenza del $\sup{A}$ è garantita dall'assioma di completezza (o di Dedekind) dei reali. Per dimostrare l'unicità, supponiamo per assurdo che il $\sup$ non sia unico ed esistano $m=\sup{A}$ $m'=\sup{A}$ con $m \neq m'$. Allora, per come è definito il $\sup$, dobbiamo avere che $m \leq m'$ e $m' \leq m$ (siccome sia $m$ e $m'$ sono dei maggioranti e, per la precisione, il minore dei maggioranti) $\implies m = m'$
	\end{myproof}	
	
\noindent Torniamo all'esercizio e definiamo, prima di procedere, la sezione di Dedekind prodotto $B(x)B(y) = \{x \in \mathbb{Q}: \exists s \in B(x), t \in B(y) : x = st \}$ e si osserva che $\forall b^s \in B(x)$ e $\forall b^t \in B(y) \implies b^s b^t = b^{s+t} \in B(x+y)$ siccome $s \leq x$ e $t \leq y$ dunque $s+t \leq x+y$ dunque $B(x)B(y) \subseteq B(x+y)$. Tuttavia si osserva che $\forall z \in \mathbb{R} : b^z \in B(x+y)$ possiamo considerare invece i numeri razionali che soddisfano la seguente proprietà $t-x < p < y$ e consideriamo a questo punto $q = t - p$ da cui avremo che $t-x < p \implies t - p < x \implies q < x$ ma allora $t = p + q$, dunque:
$$
b^t = b^{p+q} \stackrel{\text{per quanto visto sopra}}{=} b^p b^q \implies b^t \in B(x)B(y) \implies B(x+y) \subseteq B(x)B(y)
$$
In conclusione, abbiamo quindi mostrato che $B(x)B(y)=B(x+y)$. Ora però dobbiamo mostrare che $\sup{B(x+y)} = \sup{B(x)}\sup{B(y)}$: si osserva innanzitutto che $\sup{B(x+y)}=\sup{B(x)B(y)} \leq \sup{B(x)}\sup{B(y)}$. Mostriamo che il $\sup{B(x)}\sup{B(y)}$ è estremo superiore dell'insieme $B(x)B(y)$, osservando che
\begin{enumerate}[label=\protect\circled{\arabic*}]
	\item $\sup{B(x)}\sup{B(y)} \geq \sup{B(x)B(y)} \geq x \implies \sup{B(x)}\sup{B(y)} \geq x \, \forall x \in B(x)B(y)$ dunque $\sup{B(x)}\sup{B(y)}$ è maggiorante.
	\item $\forall x < \sup{B(x)}\sup{B(y)}: x$ non è un maggiorante. Per mostrare questo fatto si mostra che $\sup{B(x)B(y)} = \sup{B(x)}\sup{B(y)}$ per assurdo, supponendo (in virtù di quanto detto prima) che $\sup{B(x)B(y)} < \sup{B(x)}\sup{B(y)} \implies \frac{\sup{B(x)B(y)}}{\sup{B(y)}} < \sup{B(x)}$ e, sempre ragionando alla stessa maniera, possiamo concludere che $\frac{\sup{B(x)B(y)}}{\sup{B(x)}} < \sup{B(y)}$: si osserva che la quantità $\frac{\sup{B(x)B(y)}}{\sup{B(y)}}$ non è un maggiorante di $B(x)$ (per definizione di $\sup{B(x)}$ di cui la quantità $\frac{\sup{B(x)B(y)}}{\sup{B(y)}}$ è minore) e dunque $\exists r \in \mathbb{Q}: b^r \in B(x) :\frac{\sup{B(x)B(y)}}{\sup{B(y)}} < b^r \implies \frac{\sup{B(x)B(y)}}{b^r} < \sup{B(y)} \implies \exists s \in \mathbb{Q}: b^s \in B(y): \frac{\sup{B(x)B(y)}}{b^r} < b^s$ (perché, ragionando come prima, se la quantità $\frac{\sup{B(x)B(y)}}{b^r} < \sup{B(y)}$ deve esistere un numero razionale per cui la disuguaglianza è stretta). Ma allora si giunge ad un assurdo siccome $\sup{B(x)B(y)}=\sup{B(x+y)} < b^s b^t = b^{s+t} \in B(x+y)$ che è un assurdo.
\end{enumerate}
Dunque $\sup{B(x+y)}=\sup{B(x)}\sup{B(y)} \implies b^{x+y} = b^x b^y \, \forall x, y \in \mathbb{R}$
\end{myproof}
\begin{enumerate}[resume, label=\protect\circled{\arabic*}]
	\item Fissato $b>1, y>0$; provare che esiste un unico reale $x$ tale che $b^x = y$ utilizzando la seguente "scaletta":
		\begin{enumerate}
			\item $\forall n \in \mathbb{N}, b^n - 1 \geq n(b-1)$
			\item Dunque $b-1 \geq n(b^{\frac{1}{n}}-1)$
			\item Se $t>1$ e $n > \frac{b-1}{t-1}$ allora $b^{\frac{1}{n}}<t$
			\item Se $b^w < y$ allora $b^{w+\frac{1}{n}} < y$ per $n$ sufficientemente grande (\emph{suggerimento}: per vedere questo applicare $t=yb^{-w}$ a (c))
			\item se $b^w > y$ allora $b^{w-\frac{1}{n}}>y$ per $n$ sufficiente grande
			\item Sia $A$ l'insieme di tutti i $w$ tali che $b^w < y$ e mostrare che $x=\sup{A}$ soddisfa $b^x = y$
			\item Provare che $x$ è unico
		\end{enumerate}	
\end{enumerate}
\begin{myproof}
per mostrare la (a) possiamo usare la seguente identità e osservare che $b^i > 1 \forall i \in \mathbb{N}: i \geq 0 \implies \sum\limits_{i=0}^{n+1} b^i \geq \sum\limits_{i=0}^{n+1} 1 = n+1$
$$
b^{n+1} - 1 = (b-1)(b^n + b^{n-1} + \ldots + b + 1) \geq (b-1)(n+1)
$$
Potevamo altrimenti ragionare per induzione osservando che $b^{n+1} - 1 = (b^{n+1} + b) - (b - 1) = b(b^n - 1) - (b-1) \geq bn(b-1) - (b-1) = (b-1)(bn - 1) \geq (b-1)(n+1)$ e osservare che la tesi è banalmente vera per $n=0$. \\
Per mostrare la (b) possiamo usare il seguente lemma:
\mlenma{}{Sia $b>1$ allora $b^{\frac{1}{n}}>1 \, \forall n \in \mathbb{N}$}
\begin{myproof} Supponiamo che $b^p = \beta > 1$ con $p \in \mathbb{N}$ e per il teorema 1.21 sappiamo che esiste un unico numero $y \in \mathbb{R}$ tale che $y^p = \beta$ che indichiamo con $\beta^{\frac{1}{p}}$ dunque possiamo avere due possibili alternative: $\beta^{\frac{1}{p}} < 1$ o $\beta^{\frac{1}{p}} > 1$ ma si osserva che se per assurdo $\beta^{\frac{1}{p}} < 1 \implies (\beta^{\frac{1}{p}})^p < 1$ per gli assiomi di campo, il che è in contraddizione col fatto che $\beta > 1$.
\nt{Si osservi che $\beta^{\frac{1}{p}} \neq 1$ siccome implicherebbe che $\frac{1}{p} = 0$ il che è impossibile}
\end{myproof}
\noindent Dunque, siccome $b^{\frac{1}{n}} > 1$, vale la proprietà che abbiamo mostrato prima: ponendo $t=b^{\frac{1}{n}} \implies t^n - 1 \geq n(t-1)$ ovvero $b-1 \geq n(b^{\frac{1}{n}} - 1)$. \\
Per mostrare la (c) si osserva che se $t > 1$ e $n > \frac{b-1}{t-1}$ allora sappiamo che $b-1 \geq n(b^\frac{1}{n} - 1) \geq \frac{b-1}{t-1} (b^{\frac{1}{n}-1} - 1) \implies 1 > \frac{b^{\frac{1}{n}}-1}{t-1} \implies b^{\frac{1}{n}}-1 \leq t-1 \implies b^{\frac{1}{n}}<t$. \\
Per mostrare la (d) si osserva che se $b^w < y \implies yb^{-w} > 1$ dunque è possibile applicare la (c) utilizzando $t=yb^{-w}$ dunque $b^{\frac{1}{n}} < yb^{-w} \implies b^{\frac{1}{n}+w} < y$ naturalmente per $n > \frac{b-1}{yb^{-w}-1}$. \\
La dimostrazione per (e) è simile tuttavia cambia la "partenza", infatti la proprietà che abbiamo usato per mostrare (d) richiede che $t > 1$ e si osserva che se $b^w > y \implies \frac{b^w}{y} > 1 \implies b^{\frac{1}{n}} < \frac{b^w}{y} \implies b^{w-\frac{1}{n}} > 1$. \\ Per mostrare (f) si osserva che $A = \{w \in \mathbb{R} : b^w < y \}$ è sicuramente non vuoto e ammette un $\sup{A}$: si osserva innanzitutto che se $y>1$ allora $0 \in A$ dunque non è vuoto, se $y=1$ si osserva che $b^{-1} = \frac{1}{b} < 1 \implies -1 \in A$ e se $0 < y < 1 $ si ha che $\frac{1}{y} > 1$ e possiamo mostrare tramite il seguente lemma:
\mlenma{$y=b^x$ non è limitata superiormente}{Sia $b>1$. Allora l'insieme
$$
	\mathcal{S} = \{ b^n: n \in \mathbb{N} \}
$$
non è limitato superiormente
}
\begin{myproof}
Supponiamo per assurdo che sia limitato superiormente allora $\exists \sup{\mathcal{S}} := \alpha$ allora $\forall n \in \mathbb{N}, \alpha > b^n$. Per caratterizzazione del $\sup{\mathcal{S}}$ si ha che $\forall x < \alpha, x$ non è un maggiorante, dunque $\frac{\alpha}{b} < \alpha$ non è maggiorante, dunque esiste $k \in \mathbb{N}: \frac{\alpha}{b} < b^k \implies \alpha < b^{k+1}$ il che è assurdo 
\end{myproof}
\noindent Tramite questo lemma sappiamo che deve esistere dunque un $n \in \mathbb{N}: b^n > \frac{1}{y} \implies b^{-n} < y \implies -n \in A$. A questo punto mostriamo che $x = \sup{A}$ soddisfa $b^x=y$: supponiamo che $b^x < y \implies b^{x+\frac{1}{n}} < y \implies x + \frac{1}{n} \in A$ dunque questo contraddice la definizione di estremo superiore ($x$ non è un maggiorante); se invece $b^x > y \implies b^{x - \frac{1}{n}} > y \implies x - \frac{1}{n}$ è a sua volta un maggiorante quindi contraddice la definizione di estremo superiore (non è soddisfatta la proprietà secondo cui $\forall x < \sup{\mathcal{S}}, x$ non è maggiorante. Dunque l'unica possibilità è che $b^x = y$. \\
Per mostrare che $x$ è unico supponiamo per assurdo che $\exists x $ e $x'$ tali che $x \neq x'$ e $b^x = b^{x'} = y$. Abbiamo due possibilità: $x' > x$ oppure $x > x'$ e, senza perdita di generalità, mostriamo solamente che si giunge ad una contraddizione se $x' > x$ (la dimostrazione è equivalente nell'altro caso): si consideri il sistema
\begin{align*}
	\begin{cases}
	&b^x = y \\
	&b^{x'} = y
	\end{cases}
\end{align*}
allora dividendo membro a membro avremo che $b^{x' - x} = 1$ ma siccome $x' - x > 0 \implies b^{x' - x} > 1$, il che è assurdo (il fatto che $b^w>1$ se $w>0$ segue dal fatto che un qualunque numero razionale\footnote{nel caso dei reali si considera la sezione di Dedekind e dunque si prende un razionale appartenente alla sezione su cui valgono le considerazioni che ora sto per enunciare} positivo possa essere scritto come $\frac{m}{n}$ con $m, n \in \mathbb{N}$ e sappiamo che $b^m > 1$ e, per il lemma 1.4, concludiamo che $(b^m)^{\frac{1}{n}} > 1$).
\end{myproof}
\begin{enumerate}[resume, label=\protect\circled{\arabic*}]
	\item Mostrare che non è possibile definire una relazione d'ordine sull'insieme dei numeri complessi che renda il campo (dei numeri complessi) ordinato
\end{enumerate}
\begin{myproof}
	supponiamo per assurdo che possa essere introdotta una relazione d'ordine che renda il campo dei numeri complessi un campo ordinato, allora vuol dire che:
	\begin{enumerate}[label=\protect\circled{\arabic*}]
		\item $i > 0$
		\item $i < 0$
	\end{enumerate}
	ma allora se supponiamo che $i>0$ allora $i \cdot i > 0 \cdot i \implies -1 > 0$ ma questo contraddice completamente il fatto $1 > 0$ in un campo ordinato e dunque $-1 < 0$. Se invece $i < 0 \implies (-i) > 0 \implies (-i) \cdot (-i) > 0 \implies -1 > 0$ che è nuovamente assurdo. Naturalmente si è escluso il caso $i = 0$ per ragioni banali.
\end{myproof}
\begin{enumerate}[resume, label=\protect\circled{\arabic*}]
	\item Supponiamo che $z=a+bi$ e $w=c+di$ e definiamo su di essi una relazione d'ordine tale che $z < w$ se $a < c$ e anche se $a=c$ ma $b < d$. Mostrare che questo trasforma l'insieme dei numeri complessi in un insieme ordinato. Questo ordine possiede anche la proprietà del $\sup$?
\end{enumerate}
\begin{myproof}
Mostriamo che questa relazione d'ordine soddisfa le proprietà: dati due numeri complessi $z$ e $w$ si osserva che i numeri reali sono un campo ordinato, dunque avremo sempre che $a < c$ oppure $a > c$ oppure $a = c$ (nel primo caso avremo che $z < w$ e nel secondo caso $z > w$). L'unico caso non banale è $a = c$ che possiamo risolvere "guardando" le componenti immaginarie dei numeri complessi, che saranno a loro volta sempre dei numeri reali che, essendo ordinati, avremo sempre che $b<d$ oppure $b>d$ oppure $b=d$ (nel primo caso avremo $z < w$ e nel secondo $z > w$). Se $b=d$ allora si deve concludere che $z=w$: abbiamo quindi dimostrato che, dati due elementi $z, w \in \mathbb{C}$ vale sempre uno dei seguenti predicati:
\begin{align*}
	&z < w & &z=w & &z>w
\end{align*}
Adesso dobbiamo mostrare che se $z < w$ e $w < u$ allora $z < u$. Sia adesso $u = e+fi$ e sapendo che $z < w$ si pongono davanti a noi due possibilità:
\begin{itemize}
	\item $a < c$
	\item $a = c \wedge b < d$ 
\end{itemize}
e ragionando similmente con $w < u$ dobbiamo concludere che ci sono due sole possibilità:
\begin{itemize}
	\item $c < e$
	\item $c = e \wedge d < f$
\end{itemize}
e adesso guardiamo a tutte le possibili configurazioni (4 in totale):
\begin{enumerate}
	\item $a < c \wedge c < e \implies a < e \implies z < u$ (per la transitività della relazione d'ordine $<$)
	\item $a < c \wedge (c=e \wedge b < d) \implies a < e \implies z<u$ (siccome $c=e$)
	\item $(a=c \wedge b<d) \wedge (c < e) \implies a < e \implies z<u$ (siccome $a=c<e$)
	\item $(a=c \wedge b<d) \wedge (c=e \wedge d<f) \implies b<d<f \implies b<f \implies z<u$ 
\end{enumerate}
Per mostrare che questo insieme non possiede la proprietà dell'estremo superiore, consideriamo l'insieme definito nella seguente maniera
$$
A = \{a+bi \in \mathbb{C}: a \leq 0 \}
$$
si osserva che tutti i numeri immaginari con parte reale positiva sono dei maggioranti dunque $A$ è limitato superiormente. Supponiamo per assurdo che $\exists C \in \mathbb{C}: C \geq x \, \forall x \in A$ allora questo implica necessariamente che $\text{Re}(C) > 0$ (altrimenti non potrebbe essere un maggiorante di ogni elemento dell'insieme A) ma si osserva che fallisce miseramente la proprietà caratterizzante del $\sup$ secondo cui $\forall x < C, x$ non è maggiorante siccome qualunque numero nella forma $\frac{\text{Re}(C)}{n} + bi$ con $n \in \mathbb{N}$ sarà minore di $C$ ma sarà un maggiorante di $A$.
\end{myproof}
\begin{enumerate}[resume, label=\protect\circled{\arabic*}]
	\item Supporre che $z=a+bi$ e $w=u+iv$ e
	\begin{align*}
		&a = \left( \frac{|w|+u}{2} \right)^{\frac{1}{2}} & &b=\left(\frac{|w|-u}{2} \right)^{\frac{1}{2}}
	\end{align*}
	provare che $z^2=w$ se $v \geq 0$ e che $(\bar{z})^2 = w$ se $v \leq 0$. Concludere che ogni numero complesso (con una eccezione) ha due radici complesse
\end{enumerate}
\begin{myproof}
	calcoliamo $z^2 = (a + bi)^2 = a^2 - b^2 + 2abi = \frac{|w| + u}{2} - \frac{|w|-u}{2} + 2 \left( \frac{|w|^2 - u^2}{4} \right)^{\frac{1}{2}} = u + 2i \left( \frac{u^2 + v^2 - u^2}{4} \right)^{\frac{1}{2}} = u + i|v|$. Distinguiamo due casi:
	\begin{equation*}
		z^2 = \begin{cases}
			u+iv=w  & \text{se } v \geq 0 \\
			u-iv=\bar{w}   & \text{se } v \leq 0
		\end{cases}
	\end{equation*}
	adesso calcoliamo $(\bar{z})^2 = (a-bi)^2 = a^2 - b^2 - 2abi$ e procedendo come prima si deduce che $(\bar{z})^2 = u-|v|i$:
	\begin{equation*}
		(\bar{z})^2 = \begin{cases}
			u+iv = w & \text{se } v \leq 0 \\
			u-iv = \bar{w} & \text{se } v \geq 0
		\end{cases}
	\end{equation*}
Dunque abbiamo ottenuto quanto richiesto. Per mostrare che ogni numero complesso ha due radici si osservi che nell'esercizio abbiamo mostrato che $z$ definito come mostrato soddisfa il fatto che $z^2 = w$. Ora se $v>0$ allora si ha che le radici di $w$ sono $z$ e $-z$ mentre se $v<0$ allora si ha che le radici sono $\bar{z}$ e $-\bar{z}$. Se invece $v=0 \implies z^2 = a \implies z = \pm \sqrt{a}$ mentre se $u=v=0 \implies z=\bar{z}=0$
\end{myproof}
\begin{enumerate}[resume, label=\protect\circled{\arabic*}]
	\item Se $z$ è un numero complesso, mostrare che esiste un numero $r \geq 0$ e un numero complesso $w \in \mathbb{C}$ con $|w|=1$ tale che $z=rw$. Sono $r$ e $w$ identificati unicamente da $z$
\end{enumerate}
\begin{myproof}
	Se si considera $w=\frac{z}{|z|} \implies |w|=1$ e si considera $r=|z|$ dunque $z = rw = |z| \frac{z}{|z|} = z$ e questi sono univocamente determinati se $z \neq 0$ (in tal caso $r=0$ ma $w=a+bi \, \forall a, b \in \mathbb{R}$)
\end{myproof}
\begin{enumerate}[resume, label=\protect\circled{\arabic*}]
	\item Se $z_1, \ldots, z_n \in \mathbb{C}$ mostrare che:
	$$
	|z_1 + \ldots + z_n| \leq |z_1| + \ldots + |z_n|
	$$
\end{enumerate}
\begin{myproof}
si procede per induzione sul numero di termini, osservando che per $n=1$ e $n=2$ è banalmente vero. Mostriamo che $n \implies n+1$:
$$
|z_1 + \ldots + z_n + z_{n+1}| = |\alpha + z_{n+1}|
$$
dove con $\alpha=z_1 + \ldots + z_n$. Siccome nell'ultimo valore assoluto abbiamo solamente "due" numeri, deve valere l'ipotesi induttiva, dunque:
$$
	|\alpha + z_{n+1}| \leq |\alpha| + |z_{n+1}| \stackrel{\text{ip. induttiva}}{\leq} |z_1| + \ldots + |z_{n+1}|
$$
\end{myproof}
\begin{enumerate}[resume, label=\protect\circled{\arabic*}]
	\item Se $x, y$ sono complessi, mostrare che 
	$$
	||x|-|y|| \leq |x-y|
	$$
\end{enumerate}
\begin{myproof} si osserva che $|x| = |x-y+y| \leq |x-y| + |y| \implies |x|-|y| \leq |x-y|$ ma ragionando similmente con $|y| = |y+x-x| \leq |x-y| + |x| \implies |y|-|x| \leq |x-y|$. Siccome i reali sono ordinati dovremo avere che $|y|>|x|$ oppure $|x|>|y|$ dunque abbiamo che $||x|-|y|| = |x|-|y|$ oppure $||x|-|y||=|y|-|x|$. In ogni caso abbiamo ottenuto in entrambi i casi una relazione dove $|x-y|$ maggiora entrambe le due espressioni, dunque:
$$
	||x|-|y|| \leq |x-y|
$$
\end{myproof}
\begin{enumerate}[resume, label=\protect\circled{\arabic*}]
	\item Se $z \in \mathbb{C}$ è un numero complesso tale che $|z|=1$ (dunque $z\bar{z}=1$) calcolare
	$$
	|1+z|^2 + |1-z|^2
	$$
\end{enumerate}
\begin{myproof}
si osserva che, ponendo $z=a+bi$, $|1+z|^2 = (a+1)^2 + b^2$ mentre $|1-z|^2 = (1-a)^2 + b^2$, dunque
$$
	|1+z|^2 + |1-z|^2 = (a+1)^2 + b^2 + b^2 + (1-a)^2 = 2b^2 + (a+1)^2 + (a-1)^2 = 2a^2 + 2b^2 + 2 = 2|z|+2 = 4
$$
\end{myproof}
\begin{enumerate}[resume, label=\protect\circled{\arabic*}]
	\item Sotto quali condizioni è valida l'equazione nella disuguaglianza di Cauchy-Schwarz?
\end{enumerate}
\end{document}