\documentclass{report}
\input{preamble}
\input{macros}
\input{letterfonts}
\title{Soluzioni agli esercizi di \emph{Principles of Mathematical Analysis} di W. Rudin}
\author{Francesco Sermi}
\date{\today}


\begin{document}
	\maketitle
	\tableofcontents
	\chapter{The real and complex number system}
	\begin{enumerate}[label=\protect\circled{\arabic*}]
		\item se $r$ è razionale ($r \neq 0$) e $x$ è irrazionale, provare che $r+x$ e $rx$ sono irrazionali
		\begin{myproof}
			Supponiamo per assurdo che $r \in \mathbb{Q}$ e $x$ sia irrazionale, mentre $r+x$ e $rx$ siano razionali. Allora, $r+x = \frac{m}{n}$ con $m, n \in \mathbb{Z}$. Ma siccome $r \in \mathbb{Q} \implies \exists p, q \in \mathbb{Z}: r = \frac{p}{q}$ e
			$$
				r + x = \frac{p}{q} + x = \frac{m}{n} \implies x = \frac{p}{q} - \frac{m}{n} = \frac{pn - qm}{qn} \implies x \in \mathbb{Q}
			$$ 
			il che è assurdo. \\
			Procediamo con $rx$ alla solita maniera: se $rx \in \mathbb{Q} \implies \exists m, n \in \mathbb{Z}: rx = \frac{m}{n}$.  Ma allora, sapendo che $r = \frac{p}{q}$ con $p, q \in \mathbb{Z}$ in virtù della sua razionalità, $x = \frac{m}{nr} = \frac{mq}{np} \implies x \in \mathbb{Q}$ il che è nuovamente assurdo.
		\end{myproof}
\end{enumerate}
\begin{enumerate}[resume, label=\protect\circled{\arabic*}]
		\item Provare che non esiste razionale $q$ tale che $q^2 = 12$
\end{enumerate}
		\begin{myproof}
			si osservi il seguente lemma (di cui non daremo dimostrazione)
			\mlenma{(di Euclide)}{Sia $n \in \mathbb{Z}$ e $n$ è primo.Se $n | ab$ e $a$ è coprimo con $b$ (o viceversa) allora $n | a \vee n | b$}
		\noindent Supponiamo per assurdo che $\exists q \in \mathbb{Q} : q^2 = 12$. Data la razionalità di $q$ abbiamo che esistono $m, n \in \mathbb{Z}: q = \frac{m}{n}$ e $m$ e $n$ coprimi fra allora. Ciò implica che:
		$$
			\frac{m}{n} = \sqrt{12} \implies \frac{m^2}{n^2} = 12 \implies m^2 = 12n^2
		$$
		Questo vuol dire che $m$ è pari. Siccome $2 | m \implies \exists k \in \mathbb{Z} : m = 2k$ e dunque
		$$
		m^2 = (2k)^2 = 4k^2 = 12n^2 \implies k^2 = 3n^2
		$$
		Siccome il lato destro è divisibile per $3$ allora si deve avere che anche il lato sinistro è divisibile per $3$ e dunque, per il lemma di Euclide, si osserva che si deve avere che $3 | k \implies \exists q \in \mathbb{Z} : k = 3q$. Si deduce che
		$$
		k^2 = 9q^2 = 3n^2 \implies n^2 = 3q^2
		$$
		dunque $n^2$ è divisibile per $3$ e, sempre per il lemma di Euclide, $n$ è divisibile per $3$. Ma allora si giunge ad un assurdo siccome $m = 2k$ con $3 | k$ e $3 | n$ contro l'ipotesi di coprimità fra $m$ e $n$
		\end{myproof}
		Un'ulteriore dimostrazione poteva essere effettuata basandosi sul fatto che $\sqrt{12} = 2\sqrt{3}$ dunque, tramite l'esercizio 1, sappiamo che $rx$ è irrazionale se $x$ è irrazionale e $r$ razionale quindi la dimostrazione si riduceva a provare che $\sqrt{3}$ è irrazionale.
\begin{enumerate}[resume, label=\protect\circled{\arabic*}]
		\item Provare la seguente proposizione
\end{enumerate}		
		\mprop{Conseguenze degli assiomi moltiplicativi di cui gode il campo $\mathbb{R}$}{Gli assiomi moltiplicativi di cui gode $\mathbb{R}$ implicano le seguenti proprietà:
		\begin{enumerate}[label=\protect\circled{\arabic*}]
			\item Se $x \neq 0$ e $xy = xz \implies y=z$
			\item Se $x \neq 0$ e $xy = x \implies y = 1$
			\item Se $x \neq 0$ e $xy = 1 \implies y = x^{-1}$
			\item Se $x \neq 0$ mostrare che $(x^{-1})^{-1} = x$
		\end{enumerate}}
	\begin{myproof}
	per dimostrare la \circled{1}, banalmente, si ha che:
	$$
	y \stackrel{\text{esistenza di un elemento inverso e } x \neq 0}{=}  x x^{-1} y \stackrel{\text{prop. commutativa}}{=} xz x^{-1} \stackrel{\text{prop. commutativa}}{=} x x^{-1} z = z \implies y=z
	$$
	La \circled{2} segue direttamente dalla prima ponendo $z=1$, così come la \circled{3} ponendo $z=x^{-1}$. Per la \circled{4} si osserva che siccome $\forall x \neq 0, x x^{-1} = 1$ allora $\frac{1}{x} (\frac{1}{x})^{-1} = 1 \implies x \frac{1}{x} (\frac{1}{x})^{-1} = x \implies (\frac{1}{x})^{-1} = x$
	\end{myproof}
\begin{enumerate}[resume, label=\protect\circled{\arabic*}]
	\item Sia $E \subset A$ con $A$ insieme ordinato (totalmente? Il Rudin non ce lo fa sapere ma è abbastanza probabile). Supponiamo che $\alpha$ sia un minorante di $E$ e $\beta$ sia un maggiorante di $E$. Provare che $\alpha \leq \beta$
\end{enumerate}
\begin{myproof}
	per definizione abbiamo che se $\alpha$ è un minorante allora $\forall x \in E, \alpha \leq x$ e se $\beta$ è un maggiorante allora $\forall x \in E, x \leq \beta$. Per transitività si ha che $\alpha \leq \beta$
\end{myproof}
\begin{enumerate}[resume, label=\protect\circled{\arabic*}]
	\item Sia A un insieme non vuoto di numeri reali che è limitato inferiormente. Sia $-A$ l'insieme di tutti i numeri $-x$, con $x \in A$. Mostrare che
	$$
		\inf{A} = - \sup{(-A)}
	$$
\end{enumerate}
\begin{myproof}
	sia $y \in \mathbb{R}$ un minorante di $A$. Allora si osserva che, per definizione, $\forall x \in A, y \leq x \implies -y \geq -x$ dunque $-A$ sarà limitato superiormente. Siccome $\forall E \subset \mathbb{R} \implies \exists \sup{E}, \inf{E} \in \mathbb{R}$ allora sappiamo che $-A$ avrà $\sup{(-A)} \in \mathbb{R}$ che denoteremo con $z = \sup{(-A)}$ e mostriamo la tesi, ovvero che $\sup{(-A)} = -\inf{A}$: dobbiamo mostrare che $-z$ è l'estremo inferiore. Per farlo si osserva che se $w > -z \implies z > -w$ dunque $-w$ non è un maggiorante di $-A$ dunque $\exists y=-x(x \in A \text{ per def.}) \in -A, z > y > -w \implies -z < -y < w$ ma siccome $-y=-(-x) = x \implies x < w$ dunque $w$ non è un minorante di $A$. Se invece supponiamo esista $w \in A : w < -z \implies -w > z$ ma $-w \in -A$ il che è assurdo siccome $\nexists w \in -A: w > \sup{(-A)}$. Dunque possiamo concludere che $\inf{A} = -\sup{(-A)}$ siccome abbiamo dimostrato che:
\begin{enumerate}[label=\protect\circled{\arabic*}]
	\item $-z$ è un minorante di A;
	\item $\forall x > -z \implies x $ non è un minorante 
\end{enumerate}
\end{myproof}
\begin{enumerate}[resume, label=\protect\circled{\arabic*}]
	\item Fissato $b>1$ \begin{enumerate}
		\item Se $m, n, p, q \in \mathbb{Z}, n>0, q>0$ e $r=\frac{m}{n} = \frac{p}{q}$ mostrare che
		$$
			(b^m)^{\frac{1}{n}} = (b^p)^{\frac{1}{q}}
 		$$
 		Dunque ha senso definire $b^r = (b^m)^{\frac{1}{n}}$
 		\item Mostrare che $b^{r+s} = b^r b^s$ se $r, s \in \mathbb{Q}$
 		\item Se $x$ è reale, definiamo $B(x)$ come l'insieme di tutti i numeri $b^t$, dove $t$ è un numero razionale e $t \leq x$. Mostrare che
 		$$
 			b^r = \sup{B(r)}
 		$$
 		dunque ha senso definire
 		$$
 		b^x = \sup{B(x)}
 		$$
 		$\forall x \in \mathbb{R}$
 		\item Mostrare che $b^{x+y}=b^x b^y \, \forall x, y \in \mathbb{R}$
	\end{enumerate}
\end{enumerate}
\begin{myproof}
	Per mostrare (a) si osserva che, dal teorema 1.21, si deve avere che esistono due numeri reali $r_1$ e $r_2$ che identificano univocamente $(b^m)^{\frac{1}{n}}$ e $(b^p)^{\frac{1}{q}}$ rispettivamente. La tesi dunque si ottiene mostrando che $r_1 = r_2$. Si osserva che $(r_1)^n = b^m$ e $(r_2)^q = b^p$ e, siccome $r$ è razionale, possiamo scrivere che $m = rn$ e $p = rq$. Dunque
	\begin{align*}
		&(r_1)^n = b^{rn} & &(r_2)^q = b^{rq}
	\end{align*}
	ma elevando la prima eguaglianza da entrambi le parti per $q$ e la seconda per $n$ si ottiene che 
	\begin{align*}
		(r_1)^{nq} = b^{rnq}, (r_2)^{nq} = b^{nqr} \implies r_1 = r_2
	\end{align*}
	Per mostrare la (b) si osserva che se $r,s \in \mathbb{Q}$ allora si ha che $\exists m, n, p, q \in \mathbb{Z}: r=\frac{m}{n} \, \text{e} \, s = \frac{p}{q}$. Dunque
	$$
	b^{r+s} = b^{\frac{m}{n} + \frac{p}{q}} = b^{\frac{mq + np}{nq}}
	$$
	come prima sappiamo che esisteranno $r_1$ e $r_2$ univocamente determinati tali che $r_1 = b^{r+s}$ e $r_2 = b^{r} b^{s}$ e vogliamo mostrare che $r_1 = r_2$. Adesso si osserva che
	$$
		(b^{r+s})^{nq} = b^{mq+np} = (r_1)^{nq}
	$$
	e
	$$
		(b^{r} b^s)^{nq} = (b^{r})^{nq} (b^s)^{nq} = (b^{\frac{m}{n}})^{nq} (b^{\frac{p}{q}})^{nq} = b^{mq} b^{pn} = b^{mq+np}
	$$
	dunque $(b^{r+s})^{nq} = (b^r b^s)^{nq} \implies b^{r+s} = b^r b^s$. \\ Per mostrare la (c) bisogna innanzitutto osservare che, definendo $B(x)$ come sopra si ha che, dati $t, s \in B(x), t \leq s \implies b^t \leq b^s$. Si può mostrare questo fatto in maniera abbastanza semplice ricordando come viene definita la relazione d'ordine $\leq$ sui razionali:
\mlenma{$b^x$ è crescente con $x \in \mathbb{Q}$}{$b^x$ ristretta ai razionali è una funzione crescente}
\begin{myproof}
	Supponiamo che $s, t \in \mathbb{Q}$ con $s \leq t$. Siccome $s$ e $t$ sono razionali, allora esisteranno $m, n, p, q \in \mathbb{Z}$ tali che $s = \frac{m}{n}$ e $t = \frac{p}{q}$. Dunque abbiamo che $s \leq t \iff mq \leq np$. Abbiamo che $b^s = (b^m)^{\frac{1}{n}}$ e $b^t = (b^p)^{\frac{1}{q}}$ e sappiamo che questi numeri identificano in maniera univoca, grazie al teorema 1.21, due distinti numeri reali (tranne nel caso in cui $s = t$). Si osserva che se $r_1 = (b^m)^{\frac{1}{n}} \implies (r_1)^{n} = b^m \implies (r_1)^{nq} = b^{mq} < b^{np} = (r_2)^{nq} \implies r_1 \leq r_2$  (prendere la radice non cambia la direzione della disuguaglianza siccome possiamo sfruttare l'identità $b^n - a^n = (b-a)\sum\limits_{i=1}^{n} b^{n-i} a^{i-1}$ e osservare che $b^n \geq a^n \iff b \geq a$) 
\end{myproof}	
	
\noindent Se consideriamo $B(r)$ con $r$ razionale, allora si ha banalmente che $b^r \in B(r)$ e dev'essere l'estremo superiore: infatti, se consideriamo $t > r \implies b^t > b^r$, dunque $b^t > b^r \geq b^x \implies b^t > b^x \, \forall x: b^x \in B(r)$ dunque $t$ è un maggiorante di $B(r)$. Mostriamo che $\forall x < r: b^x$ non è un maggiorante: se per assurdo $\exists \alpha < r: b^\alpha$ è maggiorante, allora $\forall y \leq \alpha, y \leq \alpha < r \implies b^y \leq b^{\alpha} < b^r$ il che è assurdo siccome $r \in B(r)$ e abbiamo che $b^{\alpha} < b^r \leq b^{\alpha}$. \\
	Per mostrare la (d) si osserva che dobbiamo mostrare, per il punto precedente, che $\sup{B(x+y)} = \sup{B(x)}\sup{B(y)} = b^x b^y$. Per fare ciò faremo uso del seguente lemma
	\mlenma{Unicità del $\sup$ e $\inf$}{Sia $A \subseteq \mathbb{R}$ un insieme non vuoto limitato superiormente (inferiormente). Allora $\sup{A}$ ($\inf{A}$) esiste ed è unico}
	\begin{myproof}
	l'esistenza del $\sup{A}$ è garantita dall'assioma di completezza (o di Dedekind) dei reali. Per dimostrare l'unicità, supponiamo per assurdo che il $\sup$ non sia unico ed esistano $m=\sup{A}$ $m'=\sup{A}$ con $m \neq m'$. Allora, per come è definito il $\sup$, dobbiamo avere che $m \leq m'$ e $m' \leq m$ (siccome sia $m$ e $m'$ sono dei maggioranti e, per la precisione, il minore dei maggioranti) $\implies m = m'$
	\end{myproof}	
	
\noindent Torniamo all'esercizio e definiamo, prima di procedere, la sezione di Dedekind prodotto $B(x)B(y) = \{x \in \mathbb{Q}: \exists s \in B(x), t \in B(y) : x = st \}$ e si osserva che $\forall b^s \in B(x)$ e $\forall b^t \in B(y) \implies b^s b^t = b^{s+t} \in B(x+y)$ siccome $s \leq x$ e $t \leq y$ dunque $s+t \leq x+y$ dunque $B(x)B(y) \subseteq B(x+y)$. Tuttavia si osserva che $\forall z \in \mathbb{R} : b^z \in B(x+y)$ possiamo considerare invece i numeri razionali che soddisfano la seguente proprietà $t-x < p < y$ e consideriamo a questo punto $q = t - p$ da cui avremo che $t-x < p \implies t - p < x \implies q < x$ ma allora $t = p + q$, dunque:
$$
b^t = b^{p+q} \stackrel{\text{per quanto visto sopra}}{=} b^p b^q \implies b^t \in B(x)B(y) \implies B(x+y) \subseteq B(x)B(y)
$$
In conclusione, abbiamo quindi mostrato che $B(x)B(y)=B(x+y)$. Ora però dobbiamo mostrare che $\sup{B(x+y)} = \sup{B(x)}\sup{B(y)}$: si osserva innanzitutto che $\sup{B(x+y)}=\sup{B(x)B(y)} \leq \sup{B(x)}\sup{B(y)}$. Mostriamo che il $\sup{B(x)}\sup{B(y)}$ è estremo superiore dell'insieme $B(x)B(y)$, osservando che
\begin{enumerate}[label=\protect\circled{\arabic*}]
	\item $\sup{B(x)}\sup{B(y)} \geq \sup{B(x)B(y)} \geq x \implies \sup{B(x)}\sup{B(y)} \geq x \, \forall x \in B(x)B(y)$ dunque $\sup{B(x)}\sup{B(y)}$ è maggiorante.
	\item $\forall x < \sup{B(x)}\sup{B(y)}: x$ non è un maggiorante. Per mostrare questo fatto si mostra che $\sup{B(x)B(y)} = \sup{B(x)}\sup{B(y)}$ per assurdo, supponendo (in virtù di quanto detto prima) che $\sup{B(x)B(y)} < \sup{B(x)}\sup{B(y)} \implies \frac{\sup{B(x)B(y)}}{\sup{B(y)}} < \sup{B(x)}$ e, sempre ragionando alla stessa maniera, possiamo concludere che $\frac{\sup{B(x)B(y)}}{\sup{B(x)}} < \sup{B(y)}$: si osserva che la quantità $\frac{\sup{B(x)B(y)}}{\sup{B(y)}}$ non è un maggiorante di $B(x)$ (per definizione di $\sup{B(x)}$ di cui la quantità $\frac{\sup{B(x)B(y)}}{\sup{B(y)}}$ è minore) e dunque $\exists r \in \mathbb{Q}: b^r \in B(x) :\frac{\sup{B(x)B(y)}}{\sup{B(y)}} < b^r \implies \frac{\sup{B(x)B(y)}}{b^r} < \sup{B(y)} \implies \exists s \in \mathbb{Q}: b^s \in B(y): \frac{\sup{B(x)B(y)}}{b^r} < b^s$ (perché, ragionando come prima, se la quantità $\frac{\sup{B(x)B(y)}}{b^r} < \sup{B(y)}$ deve esistere un numero razionale per cui la disuguaglianza è stretta). Ma allora si giunge ad un assurdo siccome $\sup{B(x)B(y)}=\sup{B(x+y)} < b^s b^t = b^{s+t} \in B(x+y)$ che è un assurdo.
\end{enumerate}
Dunque $\sup{B(x+y)}=\sup{B(x)}\sup{B(y)} \implies b^{x+y} = b^x b^y \, \forall x, y \in \mathbb{R}$
\end{myproof}
\begin{enumerate}[resume, label=\protect\circled{\arabic*}]
	\item Fissato $b>1, y>0$; provare che esiste un unico reale $x$ tale che $b^x = y$ utilizzando la seguente "scaletta":
		\begin{enumerate}
			\item $\forall n \in \mathbb{N}, b^n - 1 \geq n(b-1)$
			\item Dunque $b-1 \geq n(b^{\frac{1}{n}}-1)$
			\item Se $t>1$ e $n > \frac{b-1}{t-1}$ allora $b^{\frac{1}{n}}<t$
			\item Se $b^w < y$ allora $b^{w+\frac{1}{n}} < y$ per $n$ sufficientemente grande (\emph{suggerimento}: per vedere questo applicare $t=yb^{-w}$ a (c))
			\item se $b^w > y$ allora $b^{w-\frac{1}{n}}>y$ per $n$ sufficiente grande
			\item Sia $A$ l'insieme di tutti i $w$ tali che $b^w < y$ e mostrare che $x=\sup{A}$ soddisfa $b^x = y$
			\item Provare che $x$ è unico
		\end{enumerate}	
\end{enumerate}
\begin{myproof}
per mostrare la (a) possiamo usare la seguente identità e osservare che $b^i > 1 \forall i \in \mathbb{N}: i \geq 0 \implies \sum\limits_{i=0}^{n+1} b^i \geq \sum\limits_{i=0}^{n+1} 1 = n+1$
$$
b^{n+1} - 1 = (b-1)(b^n + b^{n-1} + \ldots + b + 1) \geq (b-1)(n+1)
$$
Potevamo altrimenti ragionare per induzione osservando che $b^{n+1} - 1 = (b^{n+1} + b) - (b - 1) = b(b^n - 1) - (b-1) \geq bn(b-1) - (b-1) = (b-1)(bn - 1) \geq (b-1)(n+1)$ e osservare che la tesi è banalmente vera per $n=0$. \\
Per mostrare la (b) possiamo usare il seguente lemma:
\mlenma{}{Sia $b>1$ allora $b^{\frac{1}{n}}>1 \, \forall n \in \mathbb{N}$}
\begin{myproof} Supponiamo che $b^p = \beta > 1$ con $p \in \mathbb{N}$ e per il teorema 1.21 sappiamo che esiste un unico numero $y \in \mathbb{R}$ tale che $y^p = \beta$ che indichiamo con $\beta^{\frac{1}{p}}$ dunque possiamo avere due possibili alternative: $\beta^{\frac{1}{p}} < 1$ o $\beta^{\frac{1}{p}} > 1$ ma si osserva che se per assurdo $\beta^{\frac{1}{p}} < 1 \implies (\beta^{\frac{1}{p}})^p < 1$ per gli assiomi di campo, il che è in contraddizione col fatto che $\beta > 1$.
\nt{Si osservi che $\beta^{\frac{1}{p}} \neq 1$ siccome implicherebbe che $\frac{1}{p} = 0$ il che è impossibile}
\end{myproof}
\noindent Dunque, siccome $b^{\frac{1}{n}} > 1$, vale la proprietà che abbiamo mostrato prima: ponendo $t=b^{\frac{1}{n}} \implies t^n - 1 \geq n(t-1)$ ovvero $b-1 \geq n(b^{\frac{1}{n}} - 1)$. \\
Per mostrare la (c) si osserva che se $t > 1$ e $n > \frac{b-1}{t-1}$ allora sappiamo che $b-1 \geq n(b^\frac{1}{n} - 1) \geq \frac{b-1}{t-1} (b^{\frac{1}{n}-1} - 1) \implies 1 > \frac{b^{\frac{1}{n}}-1}{t-1} \implies b^{\frac{1}{n}}-1 \leq t-1 \implies b^{\frac{1}{n}}<t$. \\
Per mostrare la (d) si osserva che se $b^w < y \implies yb^{-w} > 1$ dunque è possibile applicare la (c) utilizzando $t=yb^{-w}$ dunque $b^{\frac{1}{n}} < yb^{-w} \implies b^{\frac{1}{n}+w} < y$ naturalmente per $n > \frac{b-1}{yb^{-w}-1}$. \\
La dimostrazione per (e) è simile tuttavia cambia la "partenza", infatti la proprietà che abbiamo usato per mostrare (d) richiede che $t > 1$ e si osserva che se $b^w > y \implies \frac{b^w}{y} > 1 \implies b^{\frac{1}{n}} < \frac{b^w}{y} \implies b^{w-\frac{1}{n}} > 1$. \\ Per mostrare (f) si osserva che $A = \{w \in \mathbb{R} : b^w < y \}$ è sicuramente non vuoto e ammette un $\sup{A}$: si osserva innanzitutto che se $y>1$ allora $0 \in A$ dunque non è vuoto, se $y=1$ si osserva che $b^{-1} = \frac{1}{b} < 1 \implies -1 \in A$ e se $0 < y < 1 $ si ha che $\frac{1}{y} > 1$ e possiamo mostrare tramite il seguente lemma:
\mlenma{$y=b^x$ non è limitata superiormente}{Sia $b>1$. Allora l'insieme
$$
	\mathcal{S} = \{ b^n: n \in \mathbb{N} \}
$$
non è limitato superiormente
}
\begin{myproof}
Supponiamo per assurdo che sia limitato superiormente allora $\exists \sup{\mathcal{S}} := \alpha$ allora $\forall n \in \mathbb{N}, \alpha > b^n$. Per caratterizzazione del $\sup{\mathcal{S}}$ si ha che $\forall x < \alpha, x$ non è un maggiorante, dunque $\frac{\alpha}{b} < \alpha$ non è maggiorante, dunque esiste $k \in \mathbb{N}: \frac{\alpha}{b} < b^k \implies \alpha < b^{k+1}$ il che è assurdo 
\end{myproof}
\noindent Tramite questo lemma sappiamo che deve esistere dunque un $n \in \mathbb{N}: b^n > \frac{1}{y} \implies b^{-n} < y \implies -n \in A$. A questo punto mostriamo che $x = \sup{A}$ soddisfa $b^x=y$: supponiamo che $b^x < y \implies b^{x+\frac{1}{n}} < y \implies x + \frac{1}{n} \in A$ dunque questo contraddice la definizione di estremo superiore ($x$ non è un maggiorante); se invece $b^x > y \implies b^{x - \frac{1}{n}} > y \implies x - \frac{1}{n}$ è a sua volta un maggiorante quindi contraddice la definizione di estremo superiore (non è soddisfatta la proprietà secondo cui $\forall x < \sup{\mathcal{S}}, x$ non è maggiorante. Dunque l'unica possibilità è che $b^x = y$. \\
Per mostrare che $x$ è unico supponiamo per assurdo che $\exists x $ e $x'$ tali che $x \neq x'$ e $b^x = b^{x'} = y$. Abbiamo due possibilità: $x' > x$ oppure $x > x'$ e, senza perdita di generalità, mostriamo solamente che si giunge ad una contraddizione se $x' > x$ (la dimostrazione è equivalente nell'altro caso): si consideri il sistema
\begin{align*}
	\begin{cases}
	&b^x = y \\
	&b^{x'} = y
	\end{cases}
\end{align*}
allora dividendo membro a membro avremo che $b^{x' - x} = 1$ ma siccome $x' - x > 0 \implies b^{x' - x} > 1$, il che è assurdo (il fatto che $b^w>1$ se $w>0$ segue dal fatto che un qualunque numero razionale\footnote{nel caso dei reali si considera la sezione di Dedekind e dunque si prende un razionale appartenente alla sezione su cui valgono le considerazioni che ora sto per enunciare} positivo possa essere scritto come $\frac{m}{n}$ con $m, n \in \mathbb{N}$ e sappiamo che $b^m > 1$ e, per il lemma 1.4, concludiamo che $(b^m)^{\frac{1}{n}} > 1$).
\end{myproof}
\begin{enumerate}[resume, label=\protect\circled{\arabic*}]
	\item Mostrare che non è possibile definire un ordine che trasformi un campo complesso in uno ordinato
\end{enumerate}
\begin{myproof}
	supponiamo per assurdo che possa essere introdotta una relazione d'ordine che renda il campo dei numeri complessi un campo ordinato, allora vuol dire che:
	\begin{enumerate}[label=\protect\circled{\arabic*}]
		\item $i > 0$
		\item $i < 0$
	\end{enumerate}
	ma allora se supponiamo che $i>0$ allora $i \cdot i > 0 \cdot i \implies -1 > 0$ ma questo contraddice completamente il fatto $1 > 0$ in un campo ordinato e dunque $-1 < 0$. Se invece $i < 0 \implies (-i) > 0 \implies (-i) \cdot (-i) > 0 \implies -1 > 0$ che è nuovamente assurdo. Naturalmente si è escluso il caso $i = 0$ per ragioni banali.
\end{myproof}

\end{document}